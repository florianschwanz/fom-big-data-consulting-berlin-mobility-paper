%----------------------------------
%  TextCommands
%----------------------------------
%
%
%
%
%----------------------------------
%  common textCommands
%----------------------------------
% Information: OL bedeutet ohne Leerzeichen. Damit man dieses Command z. B. vor einem Komma oder vor einem anderen Zeichen verwenden kann. Dies ist ein Best-Practis von mir und hat sich sehr bewehrt.
% Allgemein hat es sich bewert alle Wörter die man häufig schreibt und wahrscheinlich falsch oder unterscheidlich schreibt, als Textcommand zu hinterlegen.
%
%
%
\renewcommand{\symheadingname}{\langde{Symbolverzeichnis}\langen{List of Symbols}}
\newcommand{\abbreHeadingName}{\langde{Abkürzungsverzeichnis}\langen{List of Abbreviations}}
\newcommand{\headingNameInternetSources}{\langde{Internetquellen}\langen{Internet sources}}
\newcommand{\AppendixName}{\langde{Anhang}\langen{Apendix}}
\newcommand{\vglf}{\langde{Vgl.}\langen{compare}}
\newcommand{\pagef}{\langde{S. }\langen{p. }}
\newcommand{\os}{\mbox{o. S}}
\newcommand{\ojol}{\mbox{o. J.}}
\newcommand{\oj}{\ojol\ }
\newcommand{\og}{\mbox{o. g.}\ }
\newcommand{\ua}{\mbox{u. a.}\ }
\newcommand{\dah}{\mbox{d. h.}\ }
\newcommand{\zbol}{\mbox{z. B.}}
\newcommand{\zb}{\zbol\ }
\newcommand{\uamol}{unter anderem}
\newcommand{\uam}{\uamol\ }
\newcommand{\uanol}{unter anderen}%mit Leerzeichen
\newcommand{\uan}{\uanol\ }%mit Leerzeichen
\newcommand{\abbol}{Ab"-bil"-dung}
\newcommand{\abb}{\abbol\ }
\newcommand{\tabol}{Tabelle}
\newcommand{\tab}{\tabol\ }
\newcommand{\ggfol}{ggf.}
\newcommand{\ggf}{\ggfol\ }
\newcommand{\unodol}{und/oder}
\newcommand{\unod}{\unodol\ }

%----------------------------------
% project individual textCommands
%----------------------------------
\newcommand{\lehol}{Lebensmitteleinzelhandel}%Beispiel eines langen Wortes
\newcommand{\leh}{\lehol}

\newcommand{\img}[4]{\begin{figure}[H]
	\includegraphics[#3]{abbildungen/#2}
	\centering
	\caption{#4}
	\label{#1}
	\end{figure}}

