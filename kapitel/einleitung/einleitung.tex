\section{Einleitung}
Im vorliegenden Bericht werden die Ergebnisse des Projekts "Berlin Mobility - Verkehrsanalyse im Raum Berlin" aufbereitet und vorgestellt. Das Projekt wurde vom Oktober 2020 bis Februar 2021 unter Federführung der Berliner Senatsverwaltung für Umwelt, Verkehr und Klimaschutz durchgeführt. Die Senatrsverwaltung wurde bei der Projektdurchführung und der wissenschaftlichen Ausgestaltung des Vorhabens durch Experten des Fachbereichs Big Data \& Business Analytics der FOM-Hochschule für Wirtschaft und Oekonomie unterstützt. 

\subsection{Problemstellung}
Mobilität befindet sich im ständigen Wandel. Das trifft insbesondere auf urbane Mobilität in Metropolen wie London, Tokyo und Berlin zu, was auf zwei Faktoren der fortschreitenden Urbanisierung zurückzuführen ist. Aufgrund des kontinuierlichen Zuzugs ist die Einwohnerzahl Berlins in den letzten \todo[inline]{XX Jahren um XX \% gestiegen.} Gleichzeitig nimmt die Stadtfläche kontinuierlich zu. Allein zwischen 1991 und 2010 ist Berlin in der Fläche um \todo[inline]{xx \% gewachsen.} \todo[inline]{Quelle Flächenentwicklung Berlin PDF.} Beide Entwicklungen stellen für die Planung und Gestaltung einer zukunftssicheren Verkehrsinfrastruktur große Herausforderungen dar. Eine entsprechende Verkehrsplanung muss sicherstellen, dass die Mobilität heute reibungslos funktioniert und zukünftigen Mobilitätsansprüchen auf einer größeren Fläche und mit steigenden Fahrgastzahlen genügt. 

Im Juni 2018 hat das Berliner Abgeordnetenhaus zu diesem Zweck ein neues Mobilitätsgesetz beschlossen. Ziel des Berliner Mobilitätsgesetzes ist es, eine belastbare Grundlage zur Gestaltung einer zukunftsfähigen urbanen Mobilität für die Hauptstadt zu schaffen. Berlin soll mobiler, sicherer und klimafreundlicher werden. Dafür sollen Stärken aller zur Verfügung stehenden Verkehrsmittel genutzt werden. Der ÖPNV soll stark ausgebaut werden und unterschiedliche Verkehrsmittel stärker miteinander vernetzt werden. Zudem soll die Fahrradinfrastruktur deutlich verbessert werden. Durch die Summe der Maßnahmen soll die Leistungsfähigkeit des Verkehrssystems gesteigert und die Lebensqualität der Berliner erhöht werden. Das Ziel des Mobilitätsgesetzes ist es, eine Verkehrswende einzuläuten und den PKW-Verkehr bis 2050 klimaneutral zu gestalten. 

Die Umsetzung des Berliner Mobilitätsgesetztes ist wegweisend für die Berliner Mobilität der Zukunft. Investitionen die ausgehend von dem im Mobiltiätsgesetz beschriebenen Maßnahmen getätigt werden, werden die Verkehrsinfrastruktur und die Mobiltität in der Stadt langfrsitig prägen. Es ist also von zentraler Bedeutung, sicherzustellen, dass an den richtigen Stellen investiert wird und die Umsetzung effizient vorangetrieben wird. 

Bereits in den ersten zwei Jahren nach Beschlussfassung, sind die Planung und Umsetzung des Berliner Mobilitätsgesetztes regelmäßig Ziel öffentlicher Kritik. Dabei wird beanstandet, dass Maßnahmen unzureichend geplant seien und die Umsetzung nur schleppend vorangehe. \todo[inline]{Probleme hier näher beschreiben und quellen einfügen.} Zudem herrscht zwischen den agierenden politischen Akteuren weiterhin Uneinigkeit über den Stellenwert, der dem Autoverkehr zukünftig zugeschrieben werden soll.

\subsection{Projektziele}
Mit dem Projekt "Berlin Mobility - Verkehrsanalyse im Raum Berlin" verfolgt die Berliner Senatsverwaltung für Umwelt, Verkehr und Klimaschutz das Ziel ein datengetriebenes Vorgehen zur weiteren Planung und Umsetzung des Berliner Mobilitätsgesetztes zu etablieren. Das bedeutet, dass Daten für die Verkehrsanalyse sowie die Verkehrsgestaltung in der Stadt Berlin nutzbar gemacht werden sollen. 

Ausgehend von dieser übergreifenden Zielstellung leiten sich zwei konkrete Projektziele ab. Zunächst sollen mitihilfe einschlägiger Mobilitätsdaten, Geodaten und Populationsdaten Schwachstellen in der innerstädtischen Verkehrsinfrastruktur offengelegt werden. Diese Schwachstellen sollen mithilfe von Karten visuell aufgearbeitet und interaktiv erkundbar gemacht werden. Ausgehend von dieser desktirptiven Analyse sollen für besonders schlecht angebundene Orte im Stadtgebiet Lösungsansätze zur Steigerung der Anbindungsqualität entwickelt werden. 

\subsection{Projektaufbau}
Die Projektziele werden im Rahmen des Projekts in vier Arbeitspaketen realisiert. Die Arbeitspakete sin

\textbf{Arbeitspaket 1: Beschreibung der Ist-Situation}

Die Gestaltung der zukünftigen Mobilität setzt voraus, dass der Status quo bekannt ist. Der Status quo der Verkehrsinfrastruktur, umfasst zum einen die Erhebung der vorhandenen Infrastruktur, sowie ein Verständnis des Mobilitätsverhaltens der Bürger*innen als Nutzer der Infrastruktur. 

Im Rahmen des Arbeitspakets liegen Orte wie dicht besidelte Wohngebiete, Handels- und Dienstleistungszentren, Industriegebiete sowie Universitäten im Fokus der Betrachtung. Hinsichtlich der Verkehrsinfrastruktur werden alle Ebenen des öffentlichen Personennahverkehrs (ÖPNV), sowie der der Individualverkehr (PKW und Fahrrad) betrachtet. Die so gewonnen Einblicke bilden die Basis für die weitere Analyse der Mobilität in Berlin.

\textbf{Arbeitspaket 2: Aufzeigen von Schwachstellen der vorhandenen Infrastruktur}

Sobald die vorhandene Infrastruktur sowie das Mobilitätsverhalten bekannt sind, werden im Rahmen des zweiten Arbeitspakets gezielt Schwachstellen in der Infrastrukutr identifiziert. Dafür werden die  Ebenen der Verkehrsinfrastrukutr isoliert sowie kombiniert mithilfe eines auf Isochronen basierenden Analyseansatzes betrachtet. \todo[inline]{Verlinkung zu Isochronen.}

Die kombinierte Analyse der besonders im Fokus des Berliner Mobilitätsgesetzes stehenden ÖPNV-Infrastruktur (Bus, U-Bahn, S-Bahn und Straßenbahn) ermöglicht die Identifizierung von wichtigen Orten (siehe AP 1) die über eine unzureichende Verkehrsanbindung verfügen. 

\textbf{Arbeitspaket 3: Ableiten von Handlungsempfehlungen}

Die identifizierten Schwachstellen werden in Abhängigkeit der Nähe zu wichtigen Orten (siehe AP 1) priorisiert. Ausgehend davon werden Handlungsempfehlungen für fünf hoch priorisierte Orte im Stadtgebiet entwickelt. Handlungsemfephlungen umfassen konkrete (Infrastruktur-)Maßnahmen die zu einer Verbesserung der Verkehrsanbindung des infrage stehenden Ortes beitragen würden. Die Entwicklung von Maßnahmen folgt dem Persona-Ansatz um Bürger*innen-Interessen adäquat adressieren zu können.

\textbf{Arbeitspaket 4: Visuelle und Interaktive Aufbereitung der Analyseergebnisse}

Die erzielten Analyseergebnisse werden in Form einer Web-App und eines interaktiven Mobilitäts-Dashboards aufbereitet. Mit der Aufbereitung werden zwei Ziele verfolgt. Zum einen soll die visuell anspruchsvolle Aufbereitung der Analysergebnisse einen niedrigschwelligen Zugang bieten, welcher in der Bürger*innen-Kommunikation eingesetzt werden kann. Zum anderen soll das interaktive Mobilitäts-Dashboard weiterführende (Verkehrs-)Analysen ermöglichen. Das Mobilitäts-Dashboard wird so umgesetzt, dass es zu einem Prototyp für ein Planungstool von Mobilittätsmaßnahmen weiterentwickelt werden kann.

\subsection{Aufbau der Arbeit}
In Kapitel \ref{Projektumsetzung} wird das Projektvorgehen, die genutzten Daten sowie der verwendete Analyseansatz näher beschrieben. In Kapitel \ref{ergebnisse} werden die Projektergebnisse für die einzelnen Arbeitspakete vorgestellt. In Kapitel \ref{Projektumsetzung} liegt der Fokus auf der Kritischen Reflektion der Projektergebnisse in hinblick auf die Limitierung des gewählte Analyseansatzes sowie die Beschreibung der nächsten Schritte. 