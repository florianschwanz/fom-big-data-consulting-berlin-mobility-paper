\section{Einleitung}
Im vorliegenden Bericht werden die Ergebnisse des Projekts \enquote{Berlin Mobility - Verkehrsanalyse im Raum Berlin} vorgestellt. Das Projekt wurde zwischen Oktober 2020 und Februar 2021 unter Federführung der Berliner Senatsverwaltung für Umwelt, Verkehr und Klimaschutz durchgeführt. Die Senatsverwaltung wurde im Rahmen des Projektmanagements und der wissenschaftlichen Ausgestaltung des Vorhabens durch Experten des Fachbereichs Big Data \& Business Analytics der \ac{FOM} unterstützt.

\subsection{Problemstellung}\label{problems}
Mobilität befindet sich im ständigen Wandel. Das trifft insbesondere auf urbane Mobilität in Metropolen wie London, Tokyo und Berlin zu. In diesen Metropolen sind die Auswirkungen der fortschreitenden Urbanisierung sowie die stärkere Anbindung umliegender Metropolregionen besonders sichtbar. Aufgrund des kontinuierlichen Zuzugs ist beispielsweise die Einwohnerzahl Berlins zwischen 2011 und 2019 um ca. 10\% gestiegen.\footcite{Statista.2020} Gleichzeitig nimmt das Pendleraufkommen aus Brandenburg in die Hauptstadt rasant zu. Für den Zeitraum zwischen 2009 und 2019 wird ein Anstieg des Pendlerverkehrs um 26\% verzeichnet.\footcite{VBB-Pendlerblatt.2020}
Beide Entwicklungen stellen für die Planung und Gestaltung einer zukunftssicheren Verkehrsinfrastruktur große Herausforderungen dar. Eine entsprechende Verkehrsplanung muss sicherstellen, dass die Mobilität heute reibungslos funktioniert und zukünftigen Mobilitätsansprüchen in einem vergrößerten Einzugsgebiet und mit steigenden Fahrgastzahlen genügt.

Im Juni 2018 hat das Berliner Abgeordnetenhaus zu diesem Zweck ein neues Mobilitätsgesetz beschlossen. Ziel des Berliner Mobilitätsgesetzes ist es, eine belastbare Grundlage zur Gestaltung einer zukunftsfähigen urbanen Mobilität für die Hauptstadt zu schaffen.\footcite{Mobilitaetsgesetz.2020} Berlin soll mobiler, sicherer und klimafreundlicher werden. Dafür sollen die Stärken aller zur Verfügung stehenden Verkehrsmittel genutzt werden. Der \ac{ÖPNV} soll ausgebaut werden und die unterschiedlichen Verkehrsmittel sollen stärker miteinander vernetzt werden. Zudem soll die Fahrradinfrastruktur deutlich verbessert werden. Durch die Summe der Maßnahmen soll die Leistungsfähigkeit des Verkehrssystems gesteigert und die Lebensqualität der Berliner erhöht werden. Das Ziel des Mobilitätsgesetzes ist es, eine Verkehrswende einzuläuten und den innerstädtischen Verkehr bis 2050 klimaneutral zu gestalten.\footcite{Mobilitaetsgesetz.2020}

Die Umsetzung des Berliner Mobilitätsgesetztes ist wegweisend für die Berliner Mobilität der Zukunft. Investitionen die auf Basis des Mobilitätsgesetzes getätigt werden, werden die Verkehrsinfrastruktur und die Mobilität der Stadt langfristig prägen. Es ist also von zentraler Bedeutung, sicherzustellen, dass die Maßnahmen die ausgehend vom Mobilitätsgesetz definiert werden gleichermaßen zielführend und umsetzbar sind. Dies ist die Voraussetzung dafür, dass die Investitionen den Bürger*innen der Stadt auch langfristig zugutekommen und die Umsetzung effizient vorangetrieben werden kann.

Bereits in den ersten zwei Jahren nach Beschlussfassung, sind die Planung und Umsetzung des Berliner Mobilitätsgesetztes jedoch regelmäßig Ziel öffentlicher Kritik. Bürger*innen, Politiker*innen und Verbände beanstanden, dass Maßnahmen unzureichend geplant seien und die Umsetzung nur schleppend vorangehe.\footcite{Tagesspiegel.2019} Zudem wird den agierenden politischen Akteuren Uneinigkeit über den zukünftigen Stellenwert des PKW-Verkehrs vorgeworfen.\footcite{Tagesspiegel.2020}

\subsection{Projektziele}
Mit dem Projekt \enquote{Berlin Mobility - Verkehrsanalyse im Raum Berlin} verfolgt die Berliner Senatsverwaltung für Umwelt, Verkehr und Klimaschutz das Ziel, die Grundlage für ein datengetriebenes Vorgehen zur weiteren Planung und Umsetzung des Berliner Mobilitätsgesetztes zu schaffen. Das bedeutet, dass relevante Daten für die Verkehrsanalyse sowie die Verkehrsgestaltung in der Stadt Berlin nutzbar gemacht werden sollen.

Ausgehend von dieser übergreifenden Zielstellung leiten sich zwei konkrete Projektziele ab. Zunächst sollen mithilfe einschlägiger Mobilitätsdaten, Geodaten und Populationsdaten Schwachstellen in der innerstädtischen Verkehrsinfrastruktur offengelegt werden. Diese Schwachstellen sollen mithilfe von Karten visuell aufgearbeitet und interaktiv zu erkunden sein. Ausgehend von dieser deskriptiven Analyse sollen für unzureichend angebundene Orte mit hoher Priorität (siehe Kapitel~\ref{arbeitspaket_3_ableiten_von_handlungsempfehlungen}) Lösungsansätze zur Steigerung der Anbindungsqualität entwickelt werden.

\subsection{Projektaufbau}
Die Projektziele werden im Rahmen des Projekts in vier Arbeitspaketen realisiert. Die Arbeitspakete umfassen die Beschreibung der Ist-Situation, das Aufzeigen von Schwachstellen der vorhandenen Infrastruktur, das Ableiten von Handlungsempfehlungen sowie die visuelle und interaktive Aufbereitung der Analyseergebnisse. Die Arbeitspakete sind im Folgenden näher beschrieben.

\subsubsection{Arbeitspaket 1: Beschreibung der Ist-Situation}
\label{arbeitspaket_1_beschreibung_der_ist-situation}

Die Gestaltung der zukünftigen Mobilität setzt die Kenntnis des Status quo voraus. Der Status quo der Verkehrsinfrastruktur umfasst zum einen die Erhebung der vorhandenen Infrastruktur und zum anderen ein Verständnis des Mobilitätsverhaltens der Bürger*innen bei der Nutzung der Infrastruktur.

Im Rahmen des Arbeitspakets liegen wichtige Orte wie dicht besiedelte Wohngebiete, Handels- und Dienstleistungszentren, Industriegebiete sowie Universitäten im Fokus der Betrachtung. Hinsichtlich der Verkehrsinfrastruktur werden alle Ebenen des \ac{ÖPNV}, sowie die Ebene des Individualverkehrs (PKW und Fahrrad) betrachtet. Die so gewonnen Einblicke bilden die Basis für die weitere Analyse der Mobilität in Berlin.

\subsubsection{Arbeitspaket 2: Aufzeigen von Schwachstellen der vorhandenen Verkehrsinfrastruktur}

Sobald die vorhandene Infrastruktur sowie das Mobilitätsverhalten bekannt sind, werden im Rahmen des zweiten Arbeitspakets gezielt Schwachstellen in der Infrastruktur identifiziert. Dafür werden die Ebenen der Verkehrsinfrastruktur isoliert und kombiniert mithilfe eines auf Isochronen basierenden Analyseansatzes betrachtet.

Die kombinierte Analyse der besonders im Fokus des Berliner Mobilitätsgesetzes stehenden ÖPNV-Infrastruktur (Bus, U-Bahn, S-Bahn und Straßenbahn) ermöglicht die Identifizierung wichtiger Orten (siehe Kapitel~\ref{arbeitspaket_1_beschreibung_der_ist-situation}) die über eine unzureichende Verkehrsanbindung verfügen.

\subsubsection{Arbeitspaket 3: Ableiten von Handlungsempfehlungen}
\label{arbeitspaket_3_ableiten_von_handlungsempfehlungen}

Die identifizierten Schwachstellen werden in Abhängigkeit der Nähe zu wichtigen Orten (siehe Kapitel~\ref{arbeitspaket_1_beschreibung_der_ist-situation}) eingeordnet. Ausgehend davon werden Handlungsempfehlungen für fünf hoch priorisierte Orte im Stadtgebiet entwickelt. Handlungsempfehlungen umfassen konkrete (Infrastruktur-)Maßnahmen die zu einer Verbesserung der Verkehrsanbindung des infrage stehenden Ortes beitragen würden. Die Entwicklung von Maßnahmen folgt dem Persona-Ansatz um Bürger*innen-Interessen adäquat adressieren zu können.

\subsubsection{Arbeitspaket 4: Visuelle und interaktive Aufbereitung der Analyseergebnisse}

Die erzielten Analyseergebnisse werden in Form einer Webapplikation aufbereitet, welche darüber hinaus ein interaktives Mobilitäts-Dashboard beinhaltet. Mit der Aufbereitung werden zwei Ziele verfolgt. Zum einen soll die visuell anspruchsvolle Darstellung der Analyseergebnisse einen niedrigschwelligen Zugang bieten, welcher in der Bürger*innen-Kommunikation eingesetzt werden kann. Zum anderen soll das interaktive Mobilitäts-Dashboard weiterführende (Verkehrs-)Analysen ermöglichen. Das Mobilitäts-Dashboard wird so umgesetzt, dass es zu einem Prototyp für ein Planungstool von Mobilitätsmaßnahmen weiterentwickelt werden kann.

\subsection{Struktur des Berichts}
In Kapitel~\ref{projektumsetzung} werden das Projektvorgehen, die genutzten Daten sowie der verwendete Analyseansatz näher beschrieben. In Kapitel~\ref{projekt_und_analyseergebnisse} werden die Analyseergebnisse der Arbeitspakete 1 bis 3 vorgestellt. Kapitel~\ref{ergebnispraesentation} beschreibt die im Rahmen von Arbeitspaket 4 vorgenommene Aufbereitung und Bereitstellung der Analyseergebnisse als öffentlich zugängliche Webapplikation. In Kapitel~\ref{ergebnisreflexion} liegt der Fokus auf der kritischen Reflexion der Projektergebnisse. Im Zuge dessen werden die Stärken sowie Limitierungen des gewählten Analyseansatzes diskutiert. Zudem werden konkrete nächste Schritte auf dem Weg zu einer datengetriebenen Planung und Umsetzung des Berliner Mobilitätsgesetzes definiert.
