\newpage

\section{Projektumsetzung}
\label{projektumsetzung}

\subsection{Vorgehensmodell und Organisation}
\label{Vorgehensmodelle_und_organisation}

Die Umsetzung des Projekts ist unter Verwendung der agilen Methodiken \emph{Kanban} und \emph{Extreme Programming} erfolgt. Als Quellcodeverwaltungssystem ist \emph{GitHub} zum Einsatz gekommen. Die genannten Vorgehensmodelle und Werkzeuge werden im Folgenden erläutert.

\subsubsection{Kanban}
\label{kanban}

\emph{Kanban} ist ein weitverbreitetes agiles Vorgehensmodell in der Softwareentwicklung. Kanban ermöglicht durch die konsequente Visualisierung des Status einzelner Arbeitspakete eine transparente Darstellung des Projektfortschritts. Gleichzeitig verhindert die durchgehende Verfolgung des \ac{WIP}, dass das Projektteam an zu vielen Aufgaben parallel arbeitet.

\subsubsection{Extreme Programming}
\label{extreme_programming}

Bei \emph{Extreme Programming} liegt der Fokus auf der Erstellung funktionsfähiger Software zu Lasten einer formalisierten Vorgehensweise. Das Softwareprodukt wird in Iterationen weiterentwickelt, wobei stets ein lauffähiges Inkrement existiert. Da sich dieses Projekt auf das Data Storytelling fokussiert und sich damit die Implementierung auf die Erstellung eines Prototypen beschränkt, ist bewusst auf die Aufwandsschätzung einzelner Arbeitspakete verzichtet worden.

\subsubsection{Weekly Sync}
\label{weekly_sync}

Ein wöchentlicher Abstimmungstermin hat über die gesamte Projektlaufzeit Raum für die Synchronisation des Umsetzungsfortschritts, die gemeinsame Priorisierung der offenen Arbeitspakete sowie für technische Diskussionen gegeben.

\subsubsection{Pair Programming}
\label{pair_programming}

Teile der Konzeptions- und Implementierungsarbeiten sind im \emph{Pair Programming} durchgeführt worden. Bei dieser Technik teilen sich zwei Entwickler einen Bildschirm, wobei einer der beiden als Pilot und der andere als Navigator agiert. Der Pilot ist in diesem Szenario derjenige, der den Quellcode schreibt, der Navigator wiederum gibt die Richtung vor. Diese Vorgehensweise kann durch das Vier-Augen-Prinzip die Qualität des Quellcodes steigern. Implizit findet darüber hinaus auch ein Wissenstransfer zwischen den beiden Entwicklern statt.

\subsection{Werkzeuge und Projektstruktur}
\label{werkzeuge_und_projektstruktur}

Der folgende Abschnitt beschreibt die während des Projekts verwendeten Werkzeuge, welche für das Hosting der Webapplikation und das Projektmanagement genutzt worden sind.

\subsubsection{Firebase}
\label{firebase}

\emph{Firebase}\footnote{https://firebase.google.com/} bietet als sogenanntes \ac{BaaS} die Möglichkeit, Datenbanken und Webhosting als Dienstleistung zu nutzen, wodurch das Aufsetzen einer dedizierten Infrastruktur entfällt. Auf diese Weise eignet es sich insbesondere für Experimente und Prototypen. Firebase wird für das Hosting der im Rahmen des Projekts entwickelten Webapplikation verwendet.

\subsubsection{Github}
\label{github}

Das Quellcodeverwaltungssystem \emph{GitHub}\footnote{https://github.com/} dient in erster Linie zur zentralen Ablage von Quellcode innerhalb eines Projekts und ermöglicht damit die Sychronisation der Entwicklungsstände aller Entwickler eines Projekts. Hinzu kommen die Möglichkeiten, die Änderungshistorie nachzuvollziehen und Änderungen mittels Pull Requests zu reviewen. Der in diesem Projekt entwickelte Code steht unter Open-Source-Lizenzen und ist damit öffentlich einsehbar.

\begin{description}
    \item[fom-big-data-consulting-berlin-mobility-model]\footnote{https://github.com/fom-big-data/fom-big-data-consulting-berlin-mobility-model} enthält die Analytics-Komponente inklusive der verwendeten Rohdaten, sowie der für die Aufbereitung geschriebenen Python-Skripte und der Ergebnisse.
    \item[fom-big-data-consulting-berlin-mobility-frontend]\footnote{https://github.com/fom-big-data/fom-big-data-consulting-berlin-mobility-frontend} beinhaltet die Webapplikation, welche die Analyseergebnisse visuell darstellt. In Kapitel~\ref{ergebnispraesentation} wird der Aufbau der Webapplikation im Detail beschrieben.
\end{description}

\paragraph*{GitHub Projects}
\label{github_projects}

GitHub beinhaltet mit \emph{GitHub Projects}\footnote{https://github.com/features/project-management/} ein Projektmanagement-Werkzeug, welches sich zur Abbildung eines \emph{Kanban}-Workflows eignet. Arbeitspakete sind darin mit Labels versehen, um sie eindeutig verschiedenen Umsetzungsbereichen wie Webapplikation, Analysekomponente oder Texterstellung zuzuordnen.

\paragraph*{GitHub Actions}
\label{github_actions}

Mittels \emph{GitHub Actions}\footnote{https://github.com/features/actions/} lassen sich wiederkehrende Aufgaben basierend auf Änderungen an der Codebasis automatisieren. Im Rahmen dieses Projekts veranlasst beispielsweise die Veröffentlichung einer Quellcodeänderung innerhalb des Frontend-Projekts das automatische Deployment der Applikation nach Firebase, wodurch jeder Zwischenstand öffentlich sichtbar ist und evaluiert werden kann.

\subsection{Frameworks und Datenquellen}
\label{frameworks_und_datenquellen}

\subsubsection{OpenStreetMap}
\label{open_street_map}

Das quelloffene Projekt \emph{OpenStreetMap}\footnote{https://www.openstreetmap.de/} stellt Geodaten über Verkehrswege und Flächennutzung zur Verfügung, welche für die Erstellung von Kartenmaterial genutzt werden können. Die durch die OpenStreetMap Community zur Verfügung gestellten Geodaten bilden den überwiegenden Teil der im Rahmen des Projekts genutzt Datenbasis.

\subsubsection{Mapbox}
\label{mapbox}

Der Dienst \emph{Mapbox}\footnote{https://www.mapbox.com/} nutzt die von OpenStreetMap bereitgestellten Geoinformationen und stellt eine Programmierstelle zur Verfügung, um diese in eigene Applikationen einzubinden und bietet darüber hinaus die Möglichkeit, Markierungen, Animationen und zusätzliche Ebenen zu definieren. Das Layout der im Projektverlauf erstellten Karten nutzt den großen Funktionsumfang von Mapbox.

\subsubsection{Datenquellen}
\label{datenquellen}

Neben den Informationen zu Verkehrswegen und der Flächennutzung wurden folgende Datenquellen verwendet:

\begin{description}
    \item[PLZ Einwohner]\footcite{Einwohnerdaten} zur Darstellung der Einwohnerdichte auf Ebene von Postleitzahlbereichen
    \item[VBB-Fahrplandaten]\footcite{Fahrplandaten} zur Identifikation aller Haltestellen des \ac{ÖPNV}
    \item[Uber Movement - Geschwindigkeitsdaten]\footcite{Uberdaten} zur Ermittlung tatsächlich mit dem Auto gefahrener Geschwindigkeiten
\end{description}

\subsection{Methodik}
\label{methodik}

\subsubsection{Darstellung der Mobilitätsinfrastruktur durch Graphen}
\label{darstellung_der_mobiltaetsinfrastruktur_durch_graphen}

Im Kontext der Verkehrsanalyse wird die Infrastruktur mithilfe von Geodaten in Form von Graphen abgebildet, welche wiederum aus Knoten und Kanten bestehen. Als Knoten werden in diesem Kontext sowohl Haltestellen des \ac{ÖPNV} als auch Punkte des Wege- und Straßennetzes dargestellt. Kanten wiederum verbinden jeweils zwei Knoten, welche von einer Person beim Zurücklegen eines Weges genutzt werden können.

Eine zentrale Herausforderung bei der Analyse multimodaler Verkehrsrouten ist die Verbindung der Netze einzelner Verkehrsmittel. OpenStreetMap umfasst die Verkehrsnetze für jedes im Rahmen dieser Arbeit analysierte Verkehrsmittel, jedoch werden diese nur isoliert zur Verfügung gestellt. Um die Verkehrsnetze interaktiv nutzbar zu machen, sind zusätzliche Kanten notwendig, um den Wechsel von einem Verkehrsmittel zum anderen abzubilden. Das Hinzufügen der Kanten ist als Vorverarbeitungsschritt im Rahmen des Analyseprozesses implementiert worden.

\subsubsection{Isochronen als Erreichbarkeitsmetrik}
\label{isochronen_als_vergleichbarkeitsmetrik}

Als Isochrone werden sämtliche Orte bezeichnet, die ausgehend von einem gemeinsamen Ausgangspunkt in einem gegebenen Zeitraum erreicht werden können. Im Rahmen dieses Projekts kommt die Limitierung der Verkehrsmittel als zusätzliche Dimension hinzu.

In Abbildung~\ref{isochrone-20min-fom-walk} ist beispielsweise die 20-Minuten-Isochrone ausgehend vom Standort der FOM Hochschule in der Bismarkstraße in Berlin dargestellt, wobei die möglichen Strecken auf Fußwege beschränkt sind. Auffällig ist die annähernd kreisförmige Form der äußeren Hülle der Isochrone, welche darauf zurückzuführen ist, dass eine Person in einer vorgegebenen Zeit zu Fuß annähernd gleich weit in sämtliche Richtungen laufen kann.

\img{isochrone-20min-fom-walk}{isochrone-20min-fom-walk}{width=0.65\textwidth}{20-Minuten-Isochrone ausgehend vom Standort der FOM (zu Fuß)}

Im Gegensatz dazu ist die Isochrone bei der Ausweitung auf das U-Bahn-Netz, welche in Abbildung~\ref{isochrone-20min-fom-walk-subway} dargestellt ist, weiter ausgedehnt und fragmentiert. Dies kann durch naheliegende U-Bahn-Stationen und die höhere Geschwindigkeit von U-Bahnen im Vergleich zur Laufgeschwindigkeit erklärt werden. Die Isochrone erfassen demnach auch die Möglichkeit, an der nächsten U-Bahn Station auszusteigen und die weitere Strecke zu Fuß zurückzulegen.

\img{isochrone-20min-fom-walk-subway}{isochrone-20min-fom-walk-subway}{width=0.65\textwidth}{20-Minuten-Isochrone ausgehend vom Standort der FOM (zu Fuß + U-Bahn)}

Als qualitatives Maß für die Anbindungsqualität eines Punktes dient im Rahmen dieses Projekts der mittlere Radius der konvexen Außenhülle der Isochrone. Auf diese Weise wird die Reichweite betrachtet, die ausgehend von einem Ort durchschnittlich erreicht werden kann. Es wird dabei davon ausgegangen, dass die Reichweite bidirektional gültig ist. Entsprechend dieser Metrik ist ein Punkt, von welchem sich andere Orte besonders gut erreichen lassen im Umkehrschluss selbst gut zu erreichen. Die relative Güte der Konnektivität eines Punktes lässt sich im Vergleich mit anderen Punkten ermitteln.

\subsubsection{Versuchsaufbau und Vorgehensweise}
\label{versuchsaufbau_und_vorgehensweise}

Ein Ziel dieser Arbeit ist die Identifikation von Schwachstellen des öffentlichen Verkehrsnetzes der Stadt Berlin. Zu diesem Zweck sollen Orte ermittelt werden, welche entsprechend der in Kapitel~\ref{isochronen_als_vergleichbarkeitsmetrik} definierten Metrik eine relativ geringe Anbindungsqualität aufweisen.

\paragraph*{Vorbereitung}
\label{vorbereitung}

Als vorbereitender Schritt sind 10.000 zufällig über das gesamte Stadtgebiet verteilte Punkte generiert worden, wobei unbebaute Flächen, wie Parks, Wälder, Felder, Friedhöfe, sowie Wasserflächen ausgenommen worden sind. Diese generierten Punkte dienen als Ausgangspunkt für die Berechnung von Isochronen und sind somit auch Ausgangspunkt der im Rahmen des Projekts vorgenommenen Analysen. Zur Sicherstellung der Vergleichbarkeit der einzelnen Verkehrsmittel, ist ein fester Wert von 15 Minuten als Berechnungsgrundlage verwendet worden. Dieser Wert beruht auf Heuristiken und der Tatsache, dass in dieser Zeit die durchschnittliche Distanz zur nächsten ÖPNV-Haltestelle von 500 Metern\footcite{cnb} zurückgelegt werden kann. Der Wert ist wiederum bewusst auch nicht zu hoch gewählt worden, um den Fokus auf lokalen Infrastrukturproblemen zu belassen.

\paragraph*{Verkehrsmittelspezifische Analyse}
\label{verkehrsmittelspezifische_analyse}

Um die Abdeckung beziehungsweise die Schwachstellen einzelner Verkehrsmittel zu ermitteln sind die folgenden Schritte für alle untersuchten Verkehrsmittel durchgeführt worden.

\begin{itemize}
    \item Laden des Graphen des jeweiligen Verkehrsmittels via Overpass API \footnote{https://wiki.openstreetmap.org/wiki/Overpass\_API/}
    \item Laden des Graphen für Fußwege (ebenfalls via Overpass API)
    \item Kombination der beiden Graphen
    \item Hinzufügen zusätzlicher Kanten, um Umsteigemöglichkeiten abzubilden
    \item Iteration über sämtliche generierten Punkte

    \begin{itemize}
        \item Definition des generierten Punktes als Startpunkt
        \item Ermittlung sämtlicher Punkte die ausgehend vom Startpunkt in 15 Minuten erreicht werden können
        \item Berechnung der Isochrone durch Bildung der konvexen Hülle um die erreichbaren Punkte
        \item Ermittlung des durchschnittlichen Abstands zwischen Startpunkt und den Eckpunkten der konvexen Hülle
    \end{itemize}

\end{itemize}

Die Analyseergebnisse werden im GeoJSON-Format abgelegt, wobei für jeden der 10.000 analysierten Punkte neben den Koordinaten auch der Wert des durchschnittlichen Radius der Isochrone gespeichert wird.

\paragraph*{Holistische Analyse}
\label{holistische_analyse}

Um die Situation des öffentlichen Personennahverkehrs ganzheitlich beurteilen beziehungsweise präzise Verbesserungsvorschläge machen zu können, sind die oben aufgeführten Schritte ebenfalls unter Verwendung der kombinierten Verkehrsgraphen des öffentlichen Nahverkehrs durchgeführt worden. Die ganzheitliche Analyse bildet die Grundlage für den daraufbasierenden manuellen Analyseschritt. In diesen sollen diejenigen Orte mit einer schlechten Anbindung identifiziert werden, welche sich in relativer Nähe zu \acp{POI} - also beispielsweise Industrie- und Gewerbegebieten oder Universitäten und Hochschulen - befinden. Diese werden im folgenden als \emph{White Spots} bezeichnet.

\paragraph*{Personaansatz}
\label{personaansatz}

Die fünf markantesten während der holistischen Analyse identifizierten Orte sollen durch die Kreation von fünf Personas greifbarer gemacht werden. Jede der Personas bildet exemplarisch ein Problemszenario ab, welches sich jeweils auf einen der identifzierten White Spots bezieht. Neben einer persönlichen Problembeschreibung sollen auch mögliche Lösungsansätze skizziert werden.
