\newpage
\section{Reflexion der Ergebnisse} \label{fazit}
Die Ergebnisse die sich aus den Analysen ergeben sind für die aufgewendete Zeit anschaulich und repräsentativ. Diese können aber durch weitere Betrachtungen, in Hinblick auf einen Mobilitätsindex, weiter gesteigert werden.

\subsection{Lessons Learned}
Aus den Ergebnissen ist klar erkennbar, dass die innerstädtischen Wohngebiete bereits gut angebunden sind. 
Bei der Anbindung von Industrie- und Handel gibt es Handlungsbedarf. Die Verzahnung des bereits bestehenden ÖPNV Angebotes muss effizienter und effektiver gestaltet werden.

Des Weiteren ist zur besseren Vernetzung unterschiedlicher Verkehrsmittel die Entwicklung und der Ausbau von Mirkomobilitätskonzepten (bspw. Jelbi) zu prüfen und zu fördern.
Denn nur mit einem breiten diversen Angebot kann die nötige Tiefe und nachhaltige Wende in Berlin eingeleitet werden.​​

Der Plan zum Ausbau der Straßenbahn in die nord-westlichen Bezirke hat großes Potenzial. Der Senat hat ähnliche Bedarfe im Süden identifiziert, dieser Bedarf kann ausgehend von der Analyse bestätigt werden.
Wie bereits erwähnt kann damit ein breites Angebot genutzt werden. Dies wiederum führt zu einer Entlastungen in den entsprechenden Verkehrsmitteln oder Angeboten der Mobilität.

Die Außenbezirke sind häufig stark auf einzelne Linien angewiesen und eine Abhängigkeit entsteht. Die Vernetzung unterschiedlicher Verkehrsmittel ist einer der Schlüssel zur Gestaltung der Verkehrswende in Berlin. ​


\subsection{Limitierungen}
\subsubsection{Datenverfügbarkeit und Qualität}
Es konnten nicht alle Dimensionen betrachtet werden aufgrund der fehlenden Verfügbarkeit von Daten.
Die Zahl der Beschäftigten ist in den Gewerbegebieten nicht automatisiert zu ermitteln​ sowie deren aktualität konnte nicht gewährleistet werden.
Denn die Dichte bestimmt maßgeblich das zu erwartende Volumen an möglichen Fahrgästen oder Nutzern in den entsprechenden Regionen.

Die Daten zur tatsächlichen Geschwindigkeit des Autoverkehrs sind nicht für alle Straßenabschnitte verfügbar​. Diese Metriken würden eine ganzheitliche Betrachtung des Straßenverkehrs ermöglichen und würden ein vollständiges Bild zeichnen.

Zuverlässige Daten zur Auslastung des ÖPNV sind nicht verfügbar​.

Gründsätzlich wurden Wasserflächen nicht vollständig betrachtet. Darunter zählen Enzitäten wie Hausboote, gewerbliche Flächen und Verkehrsmittel auf dem Wasser.
\subsubsection{Komplexität der Analyse}
Die Dimension der Taktung des ÖPNV wird in der Analyse nicht berücksichtigt​.

Es wird kein „Penalty“ zur Nutzung des Autos eingerechnet​.

Die Barrierefreiheit von Haltestellen Stationen wird nicht berücksichtigt​.

Aktuell werden lediglich 15 Minuten Isochrone analysiert​. Aufgrund der nötigen Ressourcen musst sich für eine Ausprägung entschieden werden.


\subsubsection{Blick von außen}
Ein neutraler Blick auf die Verkehrsdaten hilft dabei Voreingenommenheit zu vermeiden und Herausforderungen aus einem neuen Blickwinkel zu betrachten.​
Analyse-Ergebnisse zeigen Schwerpunkte, die eine besonders eingeschränkte Erreichbarkeit aufweisen. Für das Validieren der Lösungsvorschläge und für die Planung neuer Verkehrswege benötigt es jedoch spezifisches Domänenwissen​

\subsection{Future Work / Folgeprojekte}

\subsubsection{Anreicherung der Analyse}
Anreicherung der Analyse durch Berücksichtigung weiterer Faktoren wie Taktung, Verkehrssicherheit, Lärmbelästigung und Klimaschutz.

\subsubsection{Finalisierung des Dashboards}
Zur Verfügungstellung eines Modus zur individuellen Verkehrsanalyse welcher Bürger*innen zur Verfügung gestellt werden kann, um die Bürgerkommunikation zu verbessern

\subsubsection{Entwicklung eines Planungstools}
Einarbeitung möglicher bzw. geplanter Streckenverläufe um Potenzial von Infrastrukturvorhaben unkompliziert und visuell evaluieren zu können.

\subsubsection{Blick von außen}
Entwicklung eines fortlaufenden Mobilitätsindex, um Veränderungen in der Anbindungsqualität feststellen zu können und um weitere Maßnahmen zu identifizieren. 

