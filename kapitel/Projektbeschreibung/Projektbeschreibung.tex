\newpage
\section{Projektbeschreibung} \label{projektbeschreibung}

\subsection{Projekthintergund}

\subsubsection{Das Berliner Mobilitätsgesetz als Startschuss der Mobilitätswende}

Mobilität befindet sich im ständigen Wandel. Das trifft insbesondere auf urbane Mobilität in rasant wachsenden Städten zu. Die Gestaltung einer urbanen Mobilität, die heute reibungslos funktioniert und auch zukünftigen Mobilitätsansprüchen genügt stellt eine besondere Herausforderung dar. Investitionsentscheidungen, die heute getroffen werden, prägen langfristig die Verkehrsinfrastruktur und somit die Mobilität innerhalb einer Stadt. Es ist also von zentraler Bedeutung, sicherzustellen, dass an den richtigen Stellen investiert wird.

Im Juni 2018 hat das Berliner Abgeordnetenhaus ein Mobilitätsgesetz beschlossen. Ziel des Berliner Mobilitätsgesetzes ist es, die Grundlage zur Gestaltung einer zukunftsfähigen urbanen Mobilität für die Hauptstadt zu schaffen. Berlin soll mobiler, sicherer und klimafreundlicher werden. Das kann nur gelingen indem die Stärken aller zur Verfügung stehenden Verkehrsmittel genutzt werden. Dafür sollen einzelne Verkehrsmittel ausgebaut werden. Zudem sollen unterschiedliche Verkehrsmittel stärker miteinander vernetzt werden, um die Leistungsfähigkeit des Verkehrssystems zu steigern und die Lebensqualität der Berliner zu erhöhen. Das Ziel des Mobilitätsgesetzes ist es, eine Verkehrswende einzuläuten und den PKW-Verkehr bis 2050 klimaneutral zu gestalten.

Zuletzt war die Umsetzung des Berliner Mobilitätsgesetztes regelmäßig Ziel öffentlicher Kritik. Dabei wird beanstandet, dass Maßnahmen unzureichend geplant seien und die Umsetzung nur schleppend vorangehe. Die Planung einer stadtweiten und zukunftssicheren Radverkehrsanlage ist beispielweise überfällig und über den Stellenwert. Zudem herrscht zwischen den agierenden politischen Akteuren weiterhin Uneinigkeit über den Stellenwert, der dem Autoverkehr zukünftig zugeschrieben werden soll.

\subsection{Projektziele}

\subsubsection{Die Nutzung von Daten für die Mobilität von morgen}

Die Gestaltung der zukünftigen Mobilität setzt voraus, dass der Status quo bekannt ist und die drängendsten Probleme identifiziert wurden. Ausgehend davon, können ganzheitliche Ideen und Konzepte entwickelt werden, die der städtischen Mobilität und vor allem den Bürger*innen langfristig zugutekommen. Eine Grundlage vielversprechender Mobilitätskonzepte liegt demnach in der Analyse der Daten, die den Status quo beschreiben.


\subsubsection{Arbeitspaket 1: Die wichtigen Datenpunkte der urbanen Mobilität identifizieren}

Die kleinste Einheit der städtischen Mobilität ist der Mensch. Aufgrund dessen ist es sinnvoll, zunächst das menschliche Mobilitätsverhalten im urbanen Raum zu betrachten, um konkrete Ansatzpunkte zur Verbesserung der Mobilität zu identifizieren. Im Folgenden geht es also darum zu verstehen, wo Menschen in Berlin leben, wo an welchen Orten sie sich häufig aufhalten und wie diese wichtigen Orte mit Straßen, Fahrradwegen und dem ÖPNV miteinander verbunden sind.

Im ersten Schritt werden dafür die Berliner Wohngebiete sowie die Einwohnerdichte auf einer Karte sichtbar gemacht. Zudem werden Orte identifiziert, von denen angenommen wird, dass sie häufig von Bürger*innen besucht werden und somit wichtige Eckpunkte der städtischen Mobilität darstellen. Dazu zählen Handels- und Dienstleistungszentren, Industriegebiete sowie Universitäten. In einem zweiten Schritt werden alle Ebenen der Verkehrsinfrastruktur mithilfe von Karten visualisiert. Die so gewonnen Einblicke bilden die Basis für die weitere Analyse der Mobilität in Berlin.

%todo: Umformulieren von auf karten gezeigt zu analysiert
