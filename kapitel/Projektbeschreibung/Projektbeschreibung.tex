\newpage
\section{Projektbeschreibung} \label{projektbeschreibung}

\subsubsection{Die Nutzung von Daten für die Mobilität von morgen}

Die Gestaltung der zukünftigen Mobilität setzt voraus, dass der Status quo bekannt ist und die drängendsten Probleme identifiziert wurden. Ausgehend davon, können ganzheitliche Ideen und Konzepte entwickelt werden, die der städtischen Mobilität und vor allem den Bürger*innen langfristig zugutekommen. Eine Grundlage vielversprechender Mobilitätskonzepte liegt demnach in der Analyse der Daten, die den Status quo beschreiben.


\subsubsection{Arbeitspaket 1: Beschreibung der Ist-Situation}

Die kleinste Einheit der städtischen Mobilität ist der Mensch. Aufgrund dessen ist es sinnvoll, zunächst das menschliche Mobilitätsverhalten im urbanen Raum zu betrachten, um konkrete Ansatzpunkte zur Verbesserung der Mobilität zu identifizieren. Im Folgenden geht es also darum zu verstehen, wo Menschen in Berlin leben, wo an welchen Orten sie sich häufig aufhalten und wie diese wichtigen Orte mit Straßen, Fahrradwegen und dem ÖPNV miteinander verbunden sind.

Im ersten Schritt werden dafür die Berliner Wohngebiete sowie die Einwohnerdichte auf einer Karte sichtbar gemacht. Zudem werden Orte identifiziert, von denen angenommen wird, dass sie häufig von Bürger*innen besucht werden und somit wichtige Eckpunkte der städtischen Mobilität darstellen. Dazu zählen Handels- und Dienstleistungszentren, Industriegebiete sowie Universitäten. In einem zweiten Schritt werden alle Ebenen der Verkehrsinfrastruktur mithilfe von Karten visualisiert. Die so gewonnen Einblicke bilden die Basis für die weitere Analyse der Mobilität in Berlin.

Notizen:
\begin{description}
    \item[$\bullet$] Identifizierung dicht besiedelter Flächen und Points-of-Interest als Eckpfeiler urbaner Mobilität
    \item[$\bullet$] Visualisierung der vorhandenen Verkehrsinfrastruktur (ÖPNV, Fahrrad, Auto) als Ausgangspunkt für weitere Analysen
  \end{description}

\subsubsection{Arbeitspaket 2: Aufzeigen von Schwachstellen der vorhandenen Infrastruktur}
Notizen: 
\begin{description}
    \item[$\bullet$] Nutzung von Isochronen zur isolierten Analyse der einzelnen Ebenen der Verkehrsinfrastruktur
    \item[$\bullet$] Verknüpfung der Ebenen zur Identifizierung von Gebieten die übergreifend am wenigsten gut angebunden sind
  \end{description} 

\subsubsection{Arbeitspaket 3: Ableiten von Handlungsempfehlungen}
Notizen:
\begin{description}
    \item[$\bullet$] Entwicklung von Empfehlungen zur Verbesserung der Verkehrssituation in den am wenigsten gut angebundenen Gebieten
    \item[$\bullet$] Beschreibung der Notwendigkeit für die Maßnahmen anhand von Personas
  \end{description} 

\subsubsection{Arbeitspaket 4: Erstellen eines Mobililitäts-Dashboards}
Notizen: 
\begin{description}
    \item[$\bullet$] Erstellung eines interaktiven Dashboards zur individuellen Erkundung der urbanen Mobilität
    \item[$\bullet$] Aufbereitung aller im Projekt verwendeten Datensätze in Interaktiven Elementen
  \end{description} 

%todo: Umformulieren von auf karten gezeigt zu analysiert
