\newpage
\section{Projekstruktur} \label{projektbeschreibung}
Die im vorangegangenen Kapitel vorgestellten Projektziele wurden im Rahmen der Prijektarbeit in vier Arbeitspaketen realisiert. Die Arbeitspakete sind in sich inhaltlich abgeschlossen und bauen aufeinander auf.

\subsection{Arbeitspaket 1: Beschreibung der Ist-Situation}
Die Gestaltung der zukünftigen Mobilität setzt voraus, dass der Status quo bekannt ist und die drängendsten Probleme identifiziert wurden. Ausgehend davon, können ganzheitliche Ideen und Konzepte entwickelt werden, die der städtischen Mobilität und vor allem den Bürger*innen langfristig zugutekommen.

Ein Verständnis des Status quo der Verkehrsinfrastruktur, setzt ein Verständnis des Mobilitätsverhaltens der Bürger*innen als Nutzer der Infrastruktur voraus. Im Rahmen dieses Arbeitspakets werden zunächst die Eckpfeiler des Mobilitätsverhaltens der Bürger*innen betrachtet. Ziel des Arbeitspakets ist es, stark frequentierte Orte sowie die zum Erreichen dieser Orte genutzte Verkehrsinfrastruktur zu identifizieren. 

Im Rahmen des Arbeitspakets liegen Orte wie dicht besidelte Wohngebiete, Handels- und Dienstleistungszentren, Industriegebiete sowie Universitäten im Fokus der Betrachtung. Hinsichtlich der Verkehrsinfrastruktur werden alle Ebenen des öffentlichen Personennahverkehrs (ÖPNV), sowie der der Individualverkehr (PKW und Fahrrad) betrachtet. Die so gewonnen Einblicke bilden die Basis für die weitere Analyse der Mobilität in Berlin.

\subsection{Arbeitspaket 2: Aufzeigen von Schwachstellen der vorhandenen Infrastruktur}
Notizen: 
\begin{description}
    \item[$\bullet$] Nutzung von Isochronen zur isolierten Analyse der einzelnen Ebenen der Verkehrsinfrastruktur
    \item[$\bullet$] Verknüpfung der Ebenen zur Identifizierung von Gebieten die übergreifend am wenigsten gut angebunden sind
  \end{description} 

\subsection{Arbeitspaket 3: Ableiten von Handlungsempfehlungen}
Notizen:
\begin{description}
    \item[$\bullet$] Entwicklung von Empfehlungen zur Verbesserung der Verkehrssituation in den am wenigsten gut angebundenen Gebieten
    \item[$\bullet$] Beschreibung der Notwendigkeit für die Maßnahmen anhand von Personas
  \end{description} 

\subsection{Arbeitspaket 4: Erstellen eines Mobililitäts-Dashboards}
Notizen: 
\begin{description}
    \item[$\bullet$] Erstellung eines interaktiven Dashboards zur individuellen Erkundung der urbanen Mobilität
    \item[$\bullet$] Aufbereitung aller im Projekt verwendeten Datensätze in Interaktiven Elementen
  \end{description} 

%todo: Umformulieren von auf karten gezeigt zu analysiert
