\newpage
\section{Projekt- /Analyseergebnisse} \label{ergebnisse}

\subsection{Arbeitspaket X: Überblick über die Berliner Verkehrsinfrastruktur}

\subsubsection{Verteilung der Wohngebiete}

Bei der Betrachtung der Berliner Wohngebiete ist auffällig, dass die Wohngebiete im Osten der Stadt deutlich großflächiger geschnitten sind als Wohngebiete in den zentralen Bezirken und im Westen der Stadt. Grundsätzlich lässt sich festhalten, dass die Wohngebiete ansonsten gleichmäßig über das Berliner Stadtgebiet verteilt sind. Diese Erkenntnis unterstützt also die Annahmen, dass Berlin eine Stadt der Kieze ist. Ausgehend von der Karte lässt sich kein klassisches Stadtzentrum erkennen.

Die bewohnte Fläche im Berliner Stadtgebiet beläuft sich auf ca. 368 Quadratkilometer. Das bedeutet, dass ca. 41 \% der gesamten Berliner Landfläche (892 Quadratkilometer) durch Wohngebiete beansprucht wird.

Die Einwohnerdichte der Berliner Bezirke im Vergleich

Durch die zusätzliche Betrachtung der Einwohnerdichte können Rückschlüsse auf die tatsächliche Verteilung von Bürger*innen im Stadtgebiet gezogen werden. Ein Blick auf die Karte zeigt, dass die zentralen Bezirke der Stadt besonders dicht besiedelt sind.

Die Flächen mit der höchsten Bevölkerungsdichte befinden sich in 10623 Charlottenburg, 12107 Mariendorf im Ortsteil Tempelhof-Schöneberg und 12623 Kaulsdorf-Mahlsdorf.

Die Wohngebiete in den Außenbezirken weisen im Vergleich zu den zentralen Bezirken zwar eine deutlich geringere Einwohnerdichte auf, machen aber aufgrund der größeren Fläche einen beträchtlichen Anteil der Einwohnerzahlen aus. Diese Dezentralität stellt eine Herausforderung für die Berliner Verkehrsinfrastruktur dar. Anders als in stark zentralisierten Städten, gilt es in Berlin die Anbindung vieler Kieze in einem großen Stadtgebiet auch untereinander sicherzustellen.



\subsubsection{Arbeitsplätze, Bildungseinrichtungen und Einkaufsmöglichkeiten als Eckpunkte der urbanen Mobilität}

Neben den dicht besiedelten Wohngebieten stellen auch Arbeitsplätze, Bildungseinrichtungen und Einkaufsmöglichkeiten Eckpfeiler der urbanen Mobilität dar. Durch die Visualisierung dieser Orte auf einer Karte wird deutlich, dass beispielsweise Handels- und Dienstleistungszentren sowie die Universitäten weitläufig über das Stadtgebiet verteilt sind. Nur wenige Bereiche, wie der Bezirk Steglitz-Zehlendorf im Südwesten der Stadt sowie die Bezirke Treptow-Köpenick und der Ortsteil Rudow im Bezirk Neukölln weisen ein besonders geringes Handels- und Dienstleistungsangebot auf. In den zentralen Bezirken findet sich wiederum ein etwas dichteres Handels- Dienstleistungsangebot.

Anders verhält es sich bei der Verteilung des Gewerbes. Große Industriegebiete wie Siemensstadt und ECONOPARK - Gewerbe und Innovationspark in Marzahn, befinden sich in den Außenbezirken. Wie in Adlershof liegen Gewerbegebiete häufiger in unmittelbarer Nähe zu Bildungseinrichtungen. Auffällig sind zudem Gebiete wie der Gewerbe und Technologiepark im Süden der Stadt, die sowohl Handel, Dienstleistung und Industrie aufweisen


\subsubsection{Das Berliner Straßennetz}

Das Straßennetz ist eines der zentralen Elemente der Berliner Verkehrsinfrastruktur. Insgesamt stehen den Berliner Bürger*innen 5.437 Kilometer öffentlicher Straßen zur Verfügung. Damit nehmen Straßen ca. 11 \% der Gesamtfläche der Stadt ein. Auf der Karte wird deutlich, dass auf ca. 80 \% des gesamten Straßennetzes (in orange) Tempolimit 30 gilt. Auf 15 \% aller Hauptverkehrsstraßen gilt Tempolimit 50 (in blau). Die weitere Infrastruktur bzw. 5 \% setzten sich aus 60er Zonen, 80er Zonen, Spielstraßen, der Stadtautobahn oder anderen begrenzt befahrbaren Flächen zusammen. Ein Blick auf die Verteilung der Tempolimits zeigt, dass ein Großteil der 30er-Zonen in Wohngebieten sowie um Schulen und Kindertagesstätten zu finden sind. Diese verkehrsberuhigten Bereiche sind durch Straßen mit einer zulässigen Höchstgeschwindigkeit von 50 km/h verbunden.

Anhand der tatsächlich gefahrenen Geschwindigkeit wird außerdem deutlich, dass die Durchschnittsgeschwindigkeit auf Berliner Straßen meist unter der zulässigen Maximalgeschwindigkeit liegt. Während der morgendlichen und abendlichen Stoßzeiten an Wochentagen, beträgt die Durchschnittsgeschwindigkeit teilweise weniger als 50 \% der zulässigen Maximalgeschwindigkeit. Die niedrigste Durchschnittsgeschwindigkeit findet sich in den Nachmittags- und Abendstunden. Daraus folgt, dass selbst das gut ausgebaute Straßennetz nicht auf den Umfang der aktuellen Verkehrslast ausgelegt ist.


\subsubsection{Der Bestand dedizierter Radverkehrsanlagen}

Aktuell existieren insgesamt 1.533 Kilometer dedizierter Radverkehrsanlagen – also Radspuren, Radwege oder Busspuren, die von Radfahrern genutzt werden bzw. für diese ausgewiesen sind. In der Stadtmitte gibt es verhältnismäßig viele Radanlagen (je nach Länge farbig auf der Karte dargestellt). In den Außenbezirken und auf kleineren Straßen teilt sich der motorisierte Verkehr und der Fahrradverkehr jedoch häufig die gleiche Fahrspur. Zudem existieren zahlreiche Unterbrechungen auf den Fahrradwegen. Die durchschnittliche Länge eines Segments bzw. einer Fahrradspur ohne Unterbrechung beträgt lediglich ca. 81 Meter. Die längste Fahrradspur ohne Unterbrechung beträgt 1,73 Kilometer.

Der Großteil der vorhandenen Fahrradwege verläuft aktuell auf Gehwegen (1.238 Kilometer). Zur Steigerung der Sicherheit wird häufig gefordert, dass diese Fahrradwege auf die Straße verlegt und verbreitert werden sollen. Zudem werden Forderungen zur Einrichtung von Radschnellwegen zum Anschluss wichtiger Orte an das Fahrradnetz laut. Beispielsweise ist der neue Hauptstadt-Flughafen BER, anders als moderne Flughäfen beispielsweise in Amsterdam, Frankfurt (Main) und Kopenhagen nicht an das Fahrradnetz angebunden.


\subsubsection{Die Berliner ÖPNV-Infrastruktur}

Die Berliner ÖPNV-Infrastruktur umfasst U-Bahnen, Straßenbahnen, S-Bahnen und Busse. Die Gesamtstrecke des ÖPNV beläuft sich auf 2.600 Kilometer Länge.

Allein die U-Bahninfrastruktur besteht aus 10 Linien die mit über 1.200 Wagen, die 173 U-Bahnhöfe im Stadtgebiet bedienen. Die U-Bahnlinien konzentrieren sich vor allem auf die zentralen Bezirke. Die Vielzahl an Linien in diesen Bezirken ermöglichen eine hohe Flexibilität durch viele Umsteigemöglichkeiten, wohingegen die Außenbezirke in der Regel nur von einzelnen Linien bedient werden.

Auf Ebene der Straßenbahn verkehren täglich 22 Linien mit 342 Fahrzeugen zwischen den 803 Haltestellen der Stadt. Hier fällt auf, dass die Straßenbahnlinien lediglich in den östlichen Bezirken verlaufen. Die Anbindung an das Straßenbahnnetz im übrigen Stadtgebiet ist nicht gegeben.

Im Gegensatz dazu bietet die Businfrastruktur eine flächendeckende Anbindung an das ÖPNV-Netz. Die 6.481 Bushaltestellen stehen gleichmäßig über das gesamte Stadtgebiet verteilt zur Verfügung. Ausgehend von den 154 Buslinien bildet die Businfrastruktur ein dichtes und weitreichendes Netz.

Die Infrastruktur der BVG wird durch 16 S-Bahnlinien ergänzt, die 166 S-Bahnhöfe bedienen und durch die VBB betrieben werden. Für die S-Bahnen stehen im Raum Berlin 335 Kilometer Schieneninfrastruktur zur Verfügung.



\subsection{Arbeitspaket 2: Analyse des Berliner Verkehrsnetzes}

\subsubsection{Die Erreichbarkeit durch den motorisierten Individualverkehr ist in Berlin weitgehend gegeben}

Die Berechnung auf Ebene des Autoverkehrs zeigt deutlich, dass die Straßeninfrastruktur in weiten Teilen der Stadt eine hohe Erreichbarkeit gewährleistet. Ausgehend von besonders gut angebundenen Orten können Bürger*innen bis zu 27 Kilometer in 15 Minuten zurücklegen (im Vergleich: Die Maximaldistanz mit dem ÖPNV beträgt 8 Kilometer im gleichen Zeitraum).

Ein Blick auf die Karte rechts im Bild zeigt, dass das Gebiet südwestlich vom Stadtzentrum stark von der Nähe zur Stadtautobahn A 100 profitiert. Bezirke wie Tempelhof-Schöneberg und Charlottenburg-Wilmersdorf weisen eine deutlich höhere Erreichbarkeit auf als die Bezirke im Osten der Stadt. Ausgehend vom Stadtteil Charlottenburg legen Autofahrer*innen durchschnittlich 15,5 Kilometer in 15 Minuten zurück.

Weniger gut angebundene Gebiete finden sich auch an den Stadträndern. Bei Kladow im Südwesten der Stadt wird die Erreichbarkeit durch den Verlauf der Havel erschwert. Auch im Norden der Stadt gibt es um den Ortsteil Französisch Buchholz Gebiete, die mit dem Auto weniger gut zu erreichen sind.


\subsubsection{Die Erreichbarkeit mit dem Fahrrad fällt im internationalen Vergleich ab}

Die Karte auf der rechten Seite visualisiert die Erreichbarkeit von Orten innerhalb des Stadtgebiets bei Nutzung der dedizierten Radinfrastruktur. Viele Berliner*innen Personen vermeiden es, aufgrund von Sicherheitsbedenken Hauptverkehrsstraßen ohne Fahrradweg zu fahren. Diese wurden daher von der Analyse ausgeschlossen. Die Analyse zeigt deutlich, dass Gebiete in den zentralen Bezirken im Vergleich zu den Orten in den Außenbezirken verhältnismäßig gut zu erreichen sind.

Ausgehend von den zentralen Bezirken legt eine Fahrradfahrerin innerhalb von 15 Minuten eine Distanz von bis zu 3,8 Kilometern zurück und erreicht damit eine Durchschnittsgeschwindigkeit von ca. 15,2 km/h. Im Vergleich zu führenden internationalen Fahrradstädten wie Kopenhagen wird deutlich, dass es hier noch großes Steigerungspotenzial gibt. In Kopenhagen beträgt die durchschnittliche Geschwindigkeit des Fahrradverkehrs 16,9 km/h. Diese Geschwindigkeit wird selbst in den besser angebundenen Gebieten der Berliner Innenstadt nicht erreicht. Die Fahrradanbindung nimmt in Richtung des Stadtrands zudem rapide ab. Lediglich die Bezirke um Hennigsdorf stellen eine Ausnahme dar. Die berechnete Erreichbarkeit einzelner Orte im Stadtgebiet, betont demnach die Lücken in der Fahrradinfrastruktur, die im ersten Abschnitt beschrieben sind.







\subsubsection{Die Businfrastruktur ist gleichmäßig ausgebaut bietet jedoch nur eine geringe Fortbewegungsgeschwindigkeit}
Während Bürger*innen mit dem Auto in 15 Minuten bis zu 27 Kilometer zurücklegen können, beträgt die maximale Distanz mit dem Bus lediglich 4,7 Kilometer. Dabei fällt auf, dass es hinsichtlich der Erreichbarkeit von Bushaltestellen im Stadtgebiet keine signifikanten Unterschiede gibt. Im Vergleich mit anderen Verkehrsmitteln bietet der Bus zwar eine gleichbleibende, aber in Bezug auf die zurücklegbare Distanz stark limitierte Anbindung. Im Durchschnitt lässt sich mit dem Bus innerhalb von 15 Minuten lediglich eine Distanz von ca. 2,9 Kilometer zurücklegen.

Die Analyse zeigt auch, dass die Anbindungsqualität in Richtung der Außenbezirke leicht abnimmt. Das kann insbesondere durch fehlende Umsteigemöglichkeiten erklärt werden. Gebiete in den Außenbezirken werden häufig nur durch einzelne Linien bedient, was sich negativ auf die Flexibilität und Erreichbarkeit auswirkt. Das wirkt insofern besonders schwer, als das es in den Außenbezirken an Alternativen mangelt. Wenn weder Bus, Straßenbahn noch U-Bahn oder S-Bahn eine zuverlässige Anbindung an den ÖPNV bieten, ist der Individualverkehr häufig die einzige Option.








\subsubsection{Bei der S-Bahn Anbindung gibt es gravierende Unterschiede}
Durch die Visualisierung der Anbindungsqualität mit der S-Bahn fällt auf, dass es ein starkes Gefälle zwischen gut und weniger gut angebundenen Orten innerhalb der Stadt gibt. Insbesondere die innerstädtischen Gebiete profitieren von der Überschneidung mehrere S-Bahnlinien. So sorgt im Ortsteil Gesundbrunnen der Zugang zu den S-Bahnlinien S1, S25, S41, S42 für eine sehr gute Anbindung. Gleiches gilt für den S-Bahnhof Friedrichstraße mit Zugang zu den Linien S3, S5, S1 und S25 sowie das Gebiet um den Bahnhof Ostkreuz mit direktem Zugang zu den Linien S3, S5, S75, S41 und S42. Ausgehend von diesen Orten können innerhalb von 15 Minuten bis zu 8 Kilometer mit der S-Bahn zurücklegen.

Im Gegensatz zu den gut angebundenen Gebieten fällt auf, dass Bürger*innen im Süden des Bezirks Neukölln auf keinerlei S-Bahninfrastruktur zurückgreifen können. Insbesondere die Ortsteile Britz, Buckow, und Rudow weisen eine schlechte Anbindung auf. Selbst in weiten Teilen des an den S-Bahn-Ring grenzenden Britz, ist keine S-Bahnstation fußläufig (d. h. innerhalb von 15 Minuten) zu erreichen. Gleiches gilt für den Ortsteil Charlottenburg-Nord sowie den Ortsteil Weißensee. Die Erreichbarkeit dieser Gebiete mit der S-Bahn wird mit der Fußgeschwindigkeit gleichgesetzt.












\subsubsection{Das U-Bahnnetz bietet lediglich in den zentralen Bezirken eine verlässliche Anbindung}
Im Gegensatz zur S-Bahn existiert hinsichtlich der U-Bahninfrastruktur, ein deutlich geringeres Gefälle zwischen den gut und den weniger gut angebundenen Gebieten. Hier ist zu berücksichtigten, dass die Berechnung der Erreichbarkeit auf der Distanz beruht, die von einem Ort durchschnittlich innerhalb von 15 Minuten zurückgelegt werden kann. Für die Berechnung der Erreichbarkeit wird zusätzlich angenommen, dass die Distanz zur nächsten U-Bahnstation in Laufgeschwindigkeit zurückgelegt wird. Es ist also davon auszugehen, dass bei einer Berechnung mit längeren Zeiträumen (bspw. 40 Minuten) ein stärkeres Gefälle auftreten würde, da der Unterschied zwischen der schnelleren Geschwindigkeit der U-Bahn und der Fußgeschwindigkeit stärker zum Tragen käme.

Die Berechnung der Anbindungsqualität zeigt, dass die U-Bahn in den zentralen Bezirken eine verlässliche Anbindung sicherstellt. Dabei sorgt beispielsweise die Linie U1 für eine hohe Interkonnektivität zwischen den U-Bahnlinien in den Bezirke Friedrichshain, Kreuzberg und Charlottenburg-Wilmersdorf. In diesen zentralen Bezirken lassen sich mit der U-Bahn innerhalb von 15 Minuten bis zu 7,2 Kilometer zurücklegen.












\subsubsection{Weite Teile der Außenbezirke sind nicht zuverlässig an das U-Bahnnetz angebunden}
Das Stadtgebiet jenseits der zentralen Bezirke ist lediglich über einzelne Linien an das U-Bahnnetz angebunden. Die Linie U3 bindet beispielsweise den Bezirk Steglitz-Zehlendorf an das U-Bahnnetz an, bietet jedoch erst an der zehnten Station die Möglichkeit zum Umstieg auf eine andere U-Bahnlinie. Eine ähnliche Situation zeigt sich auch in Hönow und Weißensee im Osten des Bezirks Pankow. Im Durchschnitt können hier 5,6 km innerhalb von 15 Minuten zurückgelegt werden. In anderen Gebieten, wie den Außenbezirken, befindet sich keine U-Bahnstation in Laufentfernung. In diesen Teilen der Stadt entspricht die Anbindungsqualität der Laufgeschwindigkeit.










\subsubsection{Weite Teile der Außenbezirke sind nicht zuverlässig an das U-Bahnnetz angebunden}
Das Stadtgebiet jenseits der zentralen Bezirke ist lediglich über einzelne Linien an das U-Bahnnetz angebunden. Die Linie U3 bindet beispielsweise den Bezirk Steglitz-Zehlendorf an das U-Bahnnetz an, bietet jedoch erst an der zehnten Station die Möglichkeit zum Umstieg auf eine andere U-Bahnlinie. Eine ähnliche Situation zeigt sich auch in Hönow und Weißensee im Osten des Bezirks Pankow. Im Durchschnitt können hier 5,6 km innerhalb von 15 Minuten zurückgelegt werden. In anderen Gebieten, wie den Außenbezirken, befindet sich keine U-Bahnstation in Laufentfernung. In diesen Teilen der Stadt entspricht die Anbindungsqualität der Laufgeschwindigkeit.






\subsubsection{Die datengetriebene Identifizierung übergreifender Infrastrukturprobleme setzt die Verknüpfung unterschiedlicher Datenquellen voraus}
Die isolierte Betrachtung einzelner Verkehrsmittel gibt spannende Einblicke in die Probleme auf den einzelnen Ebenen der Verkehrsinfrastruktur. Diese isolierte Betrachtung ist jedoch nur bedingt hilfreich, um übergreifende Herausforderungen zu identifizieren. Ziel des folgenden Abschnitts ist es, ein besseres Verständnis für die Probleme der Verkehrsinfrastruktur zu schaffen, die im Fokus der Verkehrswende gezielt ausgebaut werden soll. Demnach wird der ÖPNV – also S-Bahnen, U-Bahnen, Busse und Straßenbahnen – betrachtet. Da der Autoverkehr nicht im Fokus der Verkehrswende steht, wird die Auto-Infrastruktur an dieser Stelle nicht weiter berücksichtigt.

Für die diese verkehrsmittelübergreifende Analyse werden die Daten aller relevanten Verkehrsmittel miteinander verknüpft, um Umstiege zwischen einzelnen Verkehrsmitteln bei der Berechnung von Isochronen berücksichtigen zu können. Dies bildet die Grundlage für die datengestützte Identifizierung übergreifender Infrastrukturprobleme.







\subsubsection{Die gut angebundenen Gebiete befinden sich in den zentralen Bezirken und im Südosten der Stadt}
Ausgehend von der Betrachtung der einzelnen Ebenen ist es nicht überraschend, dass sich die Orte mit der verhältnismäßig besten Anbindung in den zentralen Bezirken befinden. Insbesondere die Bürger*innen im Ortsteil Gesundbrunnen und im angrenzenden Arnimkiez haben bereits heute Zugang zu einer engmaschigen Verkehrsinfrastruktur. Gleiches gilt für das Gebiet um die Friedrichstraße und das westliche Ende der Torstraße sowie das Gebiet um den Bahnhof Potsdamer Platz.

Auch im Südosten der Stadt ist grundsätzlich eine dichte Verkehrsinfrastruktur gegeben. Dabei stechen insbesondere die Gebiete um den Bahnhof Ostkreuz und den Bahnhof Baumschulenweg sowie das Gebiet zwischen den Bahnhöfen Neukölln und Hermannstraße hervor.








\subsubsection{Die schlechter angebundenen Gebiete befinden sich vor allem in den Außenbezirken}
Im Gegensatz zu den gut angebundenen Gebieten im Zentrum und Südosten der Stadt finden sich vor allem in den Außenbezirken Gebiete ohne ausreichende Anbindung. Beispielsweise Kladow, der südlichste Ortsteil der Stadt, ist lediglich mit dem Bus erreichbar, sodass die übergreifende Verkehrsanbindung gering ausfällt. Ähnlich steht es um die Ortsteile Müggelheim und Französisch Buchholz.

Überraschender ist, dass auch etliche zentrale Bezirke nicht über eine zuverlässige Anbindung an die Verkehrsinfrastruktur verfügen. Weder der Ortsteil Moabit, noch der Gräfekiez in Kreuzberg zeichnen sich durch eine hohe Erreichbarkeit aus. Gleiches gilt für den Ortsteil Alt-Hohenschönhausen und den gesamten Süden der Stadt jenseits des S-Bahnrings.






\subsection{Arbeitspaket X: Whitespot-Analyse}



Ausgehend von der Analyse der Verkehrsinfrastruktur können Gebiete priorisiert und Lösungsansätze entwickelt werden
Um Gebiete zu identifizieren, die im Rahmen von Infrastrukturprojekten zukünftig berücksichtigt werden sollten, werden Gebiete mit geringer Anbindungsqualität anhand ihrer Nähe zu wichtigen Orten priorisiert. Als wichtige Orte gelten Handels- und Dienstleistungszentren, Gewerbegebiete sowie Hochschulen und dicht besiedelte Wohngebiete. Ziel der Priorisierung ist es, Gebiete zu identifizieren die aufgrund der hohen Dichte wichtiger Orte über eine gute Anbindungsqualität verfügen sollten, die jedoch nicht gegeben ist. Im Rahmen der Priorisierung wurden fünf Gebiete ausgewählt. Aufbauend auf der quantitativen Analyse, die mithilfe der Isochrone durchgeführt wurde, wird die Verkehrsinfrastruktur, die zur Anbindung dieser Orte zur Verfügung steht, auch qualitativ betrachtet. Auf diese Weise wird ein tiefgreifendes Verständnis für die zugrundeliegenden Problemstellungen etabliert.

Ausgehend von den identifizierten Problemen, werden für die priorisierten Orte individuelle Lösungsansätze entwickelt, die zukünftig eine bessere Anbindung ermöglichen könnten. Dabei werden auch laufende Infrastrukturprojekte, wie der Ausbau der Straßenbahninfrastruktur im westlichen Stadtgebiet, aufgegriffen.



\subsubsection{Whitespot Spandau}
\paragraph{Industriestandort Spandau}
Im östlichen Teil Spandaus gibt es eine sehr hohe Gewerbedichte. Neben dem Ausbildungszentrum der Berliner Polizei befinden sich hier unter anderem mehrere Möbelhäuser, zwei Kraftwerke, ein Zementwerk und ein Klärwerk. Im direkten Umfeld ist die Bevölkerungsdichte relativ niedrig. Da sich hier jedoch die direkten Pendlerrouten aus dem stark bevölkerten Teil Spandaus und Falkensee schneiden, entsteht ein besonders hohes Pendleraufkommen sowohl in die Stadt hinein als auch aus der Stadt heraus. Laut dem Nahverkehrsplan des Landes Berlin befindet sich hier einer der am meisten ausgelasteten Netzbereiche.

Für den motorisierten Individualverkehr und den Busverkehr bildet der verlängerte Spandauer Damm die Hauptverkehrsader. Durch das hohe Pendleraufkommen reduziert sich jedoch die Durchschnittsgeschwindigkeit des Verkehrs gerade zu den Stoßzeiten merklich.

Trotz des hohen Verkehrsaufkommens werden weder U-Bahn noch S-Bahn wegen ihrer großen Entfernung von vielen nicht als praktikable Alternative wahrgenommen. Die nächste U-Bahnstation (Ruhleben) ist 22 Gehminuten entfernt und bildet mit der U2 eine direkte Verbindung unter anderem zum Zoologischen Garten und zum Alexanderplatz. Die nächste S-Bahnstation (Stresow) befindet sich 33 Gehminuten entfernt und bindet die S-Bahnlinien S9 und S3 sowie den Regionalbahnverkehr an. Fußläufig sind zusätzlich die Buslinien 131 und M45 erreichbar. Da die Taktung der Bus- und Bahnlinien kein schnelles Umsteigen ermöglichen, werden diese jedoch kaum genutzt, um den Weg zum Bahnverkehr zu verkürzen.

\paragraph{Lösungsansatz: Verbesserte S-Bahn Anbindung}
Um diesen Umstand zu ändern sieht der Berliner Nahverkehrsplan bereits eine Taktverdichtung des S-Bahnverkehrs auf den Linien S3 und S9 in Spandau vor. Damit die Bahnverbindungen trotz des weiten Fußwegs als Option für mehr Pendler in Frage kommen, sollten Busse eingesetzt werden, um eine direkte Verbindung zu den Bahnstationen herzustellen. Zu den Stoßzeiten sollten diese Busse regelmäßig verkehren, um die Wegzeit der letzten Meile so stark wie möglich zu verkürzen. Außerhalb der Stoßzeiten sollten die Busse als Rufbusse eingesetzt werden, um eine Adäquate Anbindung auch während des Schichtbetriebs der angrenzenden Industrie zu gewährleisten.










\subsubsection{Whitespot Charlottenburg-Nord}
\paragraph{Urban Tech Republic - TXL}
Aufgrund der Mischung aus Wohn- und Gewerbegebiet entsteht in Charlottenburg-Nord viel Pendelverkehr. Nahe dem ehemaligen Flughafen TXL sind eine große Zahl von Logistik- und Import-/Export-Unternehmen ansässig - unter Anderem DPD, DB-Schenker und TNT. Die Bewohnerzahl von Charlottenburg-Nord liegt im Stadt-Durchschnitt.

Trotz geografischer Nähe zur Ringbahn hat dieses Gebiet durch eine geringe Stationsdichte eine schlechte Anbindung an den ÖPNV. Lediglich die Buslinie 128 ist fußläufig erreichbar. Um zur nächsten S-Bahnstation (Beusselstraße) zu gelangen, muss ein Fußweg von 28 Minuten zurückgelegt werden. Zur nächsten U-Bahnstation (Jakob-Kaiser-Platz) beträgt der Fußweg sogar 30 Minuten. Die direkte Anbindung an die Stadtautobahn ermöglicht außerhalb der Stoßzeiten eine sehr gute Verbindung mit dem motorisierten Individualverkehr. Auch während des verstärkten Verkehrsaufkommens in den Morgen- und Abendstunden bietet der Individualverkehr die schnellste Verbindung.

Als besondere Herausforderung für den Verkehr in Charlottenburg Nord gilt der Ausbau des ehemaligen Flughafens TXL zum Innovations- und Entwicklungsstandort. Durch die große Anzahl neuer Arbeitsplätze, die hier entstehen sollen, wird prognostiziert, dass das Verkehrsaufkommen merklich steigen wird. Aufgrund der Begrenzung durch den Berlin-Spandauer-Schifffahrtskanal im Norden und Osten und durch den Westhafenkanal im Süden gestaltet sich die Anbindung an die vorhandene Verkehrsinfrastruktur in angrenzenden Gebieten als schwierig.


\paragraph{Lösungsansatz 1: Anbindung mittels Ruf- und Pendelbus}
Aufgrund der geografischen Einschränkungen durch die Schifffahrtskanäle in Charlottenburg Nord können bestehende Bus- oder Bahnlinien nicht so umgeplant werden, dass sinnvolle Verbindungen zu den vorhandenen Bahnstationen realisiert werden. Stattdessen würde eine Kombination aus einem Pendelbus- (während der Stoßzeiten) und einem Rufbusangebot (zwischen den Stoßzeiten) bestehende Anschlusslücken schließen. Indem die Busse bestehende Bahnstationen direkt mit den abgeschnittenen Gebieten verbinden, entsteht eine schnelle Verbindung zur Ringbahn, die die Anbindungsqualität des ÖPNV in Charlottenburg Nord signifikant steigert.

\paragraph{Lösungsansatz 2: Ausbau des Straßenbahnnetzes}
Um den zukünftigen Innovationsstandort Urban Tech Republic an das bestehende ÖPNV-Netz anzubinden, ist der Ausbau des Straßenbahnnetzes vom Kurt-Schumacher-Platz in Richtung Siemensstadt und Jakob-Kaiserplatz geplant. Mittels einer Verbindung, die über das Gebiet des ehemaligen Flugfeldes laufen soll, wird der Entwicklungsstandort direkt angebunden. Durch die Einschränkungen der Schifffahrtskanäle steigern die entstehenden Haltestellen die Mobilität für die weiteren ansässigen Gewerbe kaum, da sie ebenfalls nur mit einem Fußweg von über 15 Minuten verbunden sind.









\subsubsection{Whitespot: Landsberger Allee}
\paragraph{Gewerbe- und Industriegebiet Landsberger Allee}
Die Gegend rund um die Landsberger Allee, zwischen den Querstraßen Weißenseer Weg und Rhinstraße ist gekennzeichnet durch eine Vielzahl von Gewerbeflächen und Industrieanlagen. Zu den markantesten Einkaufsmöglichkeiten zählen das IKEA Einrichtungshaus, der Baumarkt Globus und das Dong Xuan Center. An der Landsberger Allee und den angrenzenden Querstraßen sind zudem zahlreiche Unternehmen aus den Bereichen Lebensmittelhandel, Maschinenbau und Automobilindustrie, sowie diverse Autovermietungen angesiedelt. Der öffentliche Nahverkehr ist in dieser Gegend auf senkrecht zueinander verlaufende Straßenbahnlinien beschränkt.

Eine Anbindung an die Innenstadt bieten als weitere Verkehrsmittel des öffentlichen Nahverkehrs die U-Bahn-Station Magdalenenstraße, sowie die S-Bahn-Stationen Landsberger Allee, Storkower Straße , Springpfuhl und Pölchaustraße. Allerdings sind diese nicht unmittelbar zu Fuß erreichbar. Insbesondere außerhalb der Stoßzeiten sind diese bedingt durch den geringeren Takt der Straßenbahn nur schwer zu erreichen.

\paragraph{Lösungsansätze}
Um die Anbindung an die anderen Verkehrsmittel des öffentlichen Nahverkehrs zu erhöhen, könnte zunächst eine Erhöhung der Taktung der Straßenbahn veranlasst werden. Zusätzlich könnte eine Nachtbuslinie eingerichtet werden, um die stark verringerte Taktung der Straßenbahn zwischen 24:00 und 07:00 zu kompensieren. Durch die Einrichtung eines Mobilitäts-Hubs (Jelbi Station) könnten beispielsweise Leihfahrräder zur Verfügung gestellt werden die Wegzeit zur nächsten S- oder U-Bahn-Station zu verringern. Gegebenenfalls könnte ein solcher Mobilitäts-Hub auch durch die Kooperation mehrerer ortsansässiger Unternehmen realisiert werden.










\subsubsection{Whitespot: Marzahn-Hellersdorf}
\paragraph{Der dicht besiedelte Osten Marzahn-Hellersdorfs}
Der östliche Teil Marzahn-Hellersdorfs ist von einer sehr hohen Einwohnerdichte geprägt. Die ÖPNV-Anbindung in diesem Gebiet ist stark von der Straßenbahninfrastruktur geprägt. Die Straßenbahnhaltestellen Alte-Hellerdorfer und Michendorfer bieten eine gute Anbindung innerhalb des Kiezes. Gebiete die außerhalb des Straßenbahnnetzes liegen, sind vom östlichen Rand Marzahn-Hellersdorfs jedoch kaum erreichbar. Die nächste S-Bahnstation S Mehrower Allee ist mit dem Bus 30 Minuten entfernt. Bus zur nächstgelegene U Hellersdorf sind mindestens 15 Minuten mit dem Bus oder der Tram einzuplanen.

Davon sind Arbeitnehmer und Arbeitnehmerinnen die in anderen Bezirken tätig sind besonders betroffen. Aber auch Senioren die sich innerhalb des Bezirks bewegen und beispielsweise in der Nachbarschaft engagieren, stehen vor großen Herausforderungen. Wichtige Begegnungsstätten, wie das Deutsche Rote Kreuz und der Nachbarschaftshilfe Klub 74 e.V. sind mit dem ÖPNV nicht in unter 30 Minuten zu erreichen. Weitere Wege, beispielsweise bis zum Alexanderplatz, sind zwar mit der Straßenbahn erreichbar, setzen aber eine Reisezeit von mindestens 45 Minuten voraus.

\paragraph{Lösungsansätze}
Um die erreichbarkeit innerhalb des Bezirks weiter zu stärken, sollte die Taktung der bestehenden Bus- und Straßenbahn Linien weiter erhöht werden. Um die Anbindung an das übrige Stadtgebiet zu stärken, sollte die Anbindung an die S Mehrower Allee und die U Hellersdorf verbessert werden. Zu diesem Zweck könnten Expressbus-Linien eingerichtet werden, die die bestehende ÖPNV-Infrastruktur besser verknüpft.








\subsubsection{Whitespot: Britz}
\paragraph{Gewerbegebiet Gradestraße}
Anhand der genaueren Betrachtung von Gewerbedichte und Lücken innerhalb der Verkehrsinfrastruktur konnte das Gewerbegebiet an der Gradestraße, Ecke Tempelhofer Weg identifiziert werden. In diesem Gewerbegebiet im nördlichen Britz, sind international agierende Unternehmen und wichtige Arbeitgeber wie Kieback&Peter GmbH, Linde Gas Deutschland und Terra Naturkost Handels KG angesiedelt. Auch Unternehmen wie Konica Minolta Business Solutions Deutschland GmbH haben hier einen Standort. Zusätzlich zu dieser hohen Industrie- und Gewerbedichte, weist das Gebiet eine verhältnismäßig hohe Einwohnerdichte auf. Die Verkehrsanbindung des Gewerbegebiets Gradestraße ist jedoch mangelhaft.

Die Anbindung des Gewerbegebiets Gradestraße hängt stark von der guten Erreichbarkeit der Stadtautobahn ab. Die Anbindung an den ÖPNV ist stark limitiert. Ausgehend von dem Gewerbegebiet sind die Buslinien 170 und M44 fußläufig erreichbar. Die Buslinie 170 wird außerhalb der Stoßzeiten nur im 20 Minuten-Takt bedient und selbst die regelmäßiger verkehrende Linie M44 bietet keine ausreichende Anbindung an den ÖPNV. Die U-Bahnstationen Blaschkoallee und Ullsteinstraße sind je 26 Gehminuten und 9 Bus-Minuten entfernt. Die nächste S-Bahnstation (Hermannstraße) ist sogar 32 Gehminuten entfernt. Selbst mit dem Bus M44 ist für die S-Bahnstation Hermannstraße, aufgrund von vier Zwischenhalten, im besten Fall in 17 Minuten zu rechnen.

\paragraph{Lösungsansatz 1: Erhöhte Taktung und neue Buslinie}
Aktuell gibt es in Britz und dem Raum Gradestraße keine laufenden oder geplanten Infrastrukturprojekte. Damit die Anbindung des Gewerbegebiets verbessert werden kann, sollte zunächst die Erreichbarkeit der bestehenden ÖPNV-Infrastruktur verbessert werden. Durch eine erhöhte Taktung des Busverkehrs außerhalb der Stoßzeiten wäre es möglich, dass im Schichtsystem tätige Arbeitnehmer*innen den ÖPNV anstelle des Individualverkehrs nutzen können. Zusätzlich könnte die Einführung einer neuen “Express-Linie” in Richtung der S-Bahnstation Hermannstraße die Wegzeit deutlich verkürzen. Auf diese Weise könnte die Attraktivität des ÖPNV deutlich gesteigert werden.

\paragraph{Lösungsansatz 2: Ausbau des Straßenbahnnetzes}
Im Rahmen der generellen Stärkung des ÖPNV-Netzes, die der Nahverkehrsplan des Landes Berlin in den kommenden Jahren vorsieht, könnte der Ausbau des Straßenbahn-Netzes im Süden des Bezirks Neukölln zu einer deutlichen Entspannung der Verkehrssituation beitragen. Dies ist bereits als Bedarf in den längerfristigen Planungen aufgenommen und könnte voraussichtlich ab 2023 umgesetzt werden.
