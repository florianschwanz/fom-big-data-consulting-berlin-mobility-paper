\newpage
\section{Projektergebnisse} \label{infos}

\subsection{Inhaltliche Analyseergebnisse}

\subsubsection{Arbeitspaket 1: Analyse des Berliner Verkehrsnetzes}

\paragraph{Verteilung der Wohngebiete}

Bei der Betrachtung der Berliner Wohngebiete ist auffällig, dass die Wohngebiete im Osten der Stadt deutlich großflächiger geschnitten sind als Wohngebiete in den zentralen Bezirken und im Westen der Stadt. Grundsätzlich lässt sich festhalten, dass die Wohngebiete ansonsten gleichmäßig über das Berliner Stadtgebiet verteilt sind. Diese Erkenntnis unterstützt also die Annahmen, dass Berlin eine Stadt der Kieze ist. Ausgehend von der Karte lässt sich kein klassisches Stadtzentrum erkennen.

Die bewohnte Fläche im Berliner Stadtgebiet beläuft sich auf ca. 368 Quadratkilometer. Das bedeutet, dass ca. 41 \% der gesamten Berliner Landfläche (892 Quadratkilometer) durch Wohngebiete beansprucht wird.

Die Einwohnerdichte der Berliner Bezirke im Vergleich

Durch die zusätzliche Betrachtung der Einwohnerdichte können Rückschlüsse auf die tatsächliche Verteilung von Bürger*innen im Stadtgebiet gezogen werden. Ein Blick auf die Karte zeigt, dass die zentralen Bezirke der Stadt besonders dicht besiedelt sind.

Die Flächen mit der höchsten Bevölkerungsdichte befinden sich in 10623 Charlottenburg, 12107 Mariendorf im Ortsteil Tempelhof-Schöneberg und 12623 Kaulsdorf-Mahlsdorf.

Die Wohngebiete in den Außenbezirken weisen im Vergleich zu den zentralen Bezirken zwar eine deutlich geringere Einwohnerdichte auf, machen aber aufgrund der größeren Fläche einen beträchtlichen Anteil der Einwohnerzahlen aus. Diese Dezentralität stellt eine Herausforderung für die Berliner Verkehrsinfrastruktur dar. Anders als in stark zentralisierten Städten, gilt es in Berlin die Anbindung vieler Kieze in einem großen Stadtgebiet auch untereinander sicherzustellen.



\paragraph{Arbeitsplätze, Bildungseinrichtungen und Einkaufsmöglichkeiten als Eckpunkte der urbanen Mobilität}

Neben den dicht besiedelten Wohngebieten stellen auch Arbeitsplätze, Bildungseinrichtungen und Einkaufsmöglichkeiten Eckpfeiler der urbanen Mobilität dar. Durch die Visualisierung dieser Orte auf einer Karte wird deutlich, dass beispielsweise Handels- und Dienstleistungszentren sowie die Universitäten weitläufig über das Stadtgebiet verteilt sind. Nur wenige Bereiche, wie der Bezirk Steglitz-Zehlendorf im Südwesten der Stadt sowie die Bezirke Treptow-Köpenick und der Ortsteil Rudow im Bezirk Neukölln weisen ein besonders geringes Handels- und Dienstleistungsangebot auf. In den zentralen Bezirken findet sich wiederum ein etwas dichteres Handels- Dienstleistungsangebot.

Anders verhält es sich bei der Verteilung des Gewerbes. Große Industriegebiete wie Siemensstadt und ECONOPARK - Gewerbe und Innovationspark in Marzahn, befinden sich in den Außenbezirken. Wie in Adlershof liegen Gewerbegebiete häufiger in unmittelbarer Nähe zu Bildungseinrichtungen. Auffällig sind zudem Gebiete wie der Gewerbe und Technologiepark im Süden der Stadt, die sowohl Handel, Dienstleistung und Industrie aufweisen


\paragraph{Das Berliner Straßennetz}

Das Straßennetz ist eines der zentralen Elemente der Berliner Verkehrsinfrastruktur. Insgesamt stehen den Berliner Bürger*innen 5.437 Kilometer öffentlicher Straßen zur Verfügung. Damit nehmen Straßen ca. 11 \% der Gesamtfläche der Stadt ein. Auf der Karte wird deutlich, dass auf ca. 80 \% des gesamten Straßennetzes (in orange) Tempolimit 30 gilt. Auf 15 \% aller Hauptverkehrsstraßen gilt Tempolimit 50 (in blau). Die weitere Infrastruktur bzw. 5 \% setzten sich aus 60er Zonen, 80er Zonen, Spielstraßen, der Stadtautobahn oder anderen begrenzt befahrbaren Flächen zusammen. Ein Blick auf die Verteilung der Tempolimits zeigt, dass ein Großteil der 30er-Zonen in Wohngebieten sowie um Schulen und Kindertagesstätten zu finden sind. Diese verkehrsberuhigten Bereiche sind durch Straßen mit einer zulässigen Höchstgeschwindigkeit von 50 km/h verbunden.

Anhand der tatsächlich gefahrenen Geschwindigkeit wird außerdem deutlich, dass die Durchschnittsgeschwindigkeit auf Berliner Straßen meist unter der zulässigen Maximalgeschwindigkeit liegt. Während der morgendlichen und abendlichen Stoßzeiten an Wochentagen, beträgt die Durchschnittsgeschwindigkeit teilweise weniger als 50 \% der zulässigen Maximalgeschwindigkeit. Die niedrigste Durchschnittsgeschwindigkeit findet sich in den Nachmittags- und Abendstunden. Daraus folgt, dass selbst das gut ausgebaute Straßennetz nicht auf den Umfang der aktuellen Verkehrslast ausgelegt ist.


\paragraph{Der Bestand dedizierter Radverkehrsanlagen}

Aktuell existieren insgesamt 1.533 Kilometer dedizierter Radverkehrsanlagen – also Radspuren, Radwege oder Busspuren, die von Radfahrern genutzt werden bzw. für diese ausgewiesen sind. In der Stadtmitte gibt es verhältnismäßig viele Radanlagen (je nach Länge farbig auf der Karte dargestellt). In den Außenbezirken und auf kleineren Straßen teilt sich der motorisierte Verkehr und der Fahrradverkehr jedoch häufig die gleiche Fahrspur. Zudem existieren zahlreiche Unterbrechungen auf den Fahrradwegen. Die durchschnittliche Länge eines Segments bzw. einer Fahrradspur ohne Unterbrechung beträgt lediglich ca. 81 Meter. Die längste Fahrradspur ohne Unterbrechung beträgt 1,73 Kilometer.

Der Großteil der vorhandenen Fahrradwege verläuft aktuell auf Gehwegen (1.238 Kilometer). Zur Steigerung der Sicherheit wird häufig gefordert, dass diese Fahrradwege auf die Straße verlegt und verbreitert werden sollen. Zudem werden Forderungen zur Einrichtung von Radschnellwegen zum Anschluss wichtiger Orte an das Fahrradnetz laut. Beispielsweise ist der neue Hauptstadt-Flughafen BER, anders als moderne Flughäfen beispielsweise in Amsterdam, Frankfurt (Main) und Kopenhagen nicht an das Fahrradnetz angebunden.


\paragraph{Die Berliner ÖPNV-Infrastruktur}

Die Berliner ÖPNV-Infrastruktur umfasst U-Bahnen, Straßenbahnen, S-Bahnen und Busse. Die Gesamtstrecke des ÖPNV beläuft sich auf 2.600 Kilometer Länge.

Allein die U-Bahninfrastruktur besteht aus 10 Linien die mit über 1.200 Wagen, die 173 U-Bahnhöfe im Stadtgebiet bedienen. Die U-Bahnlinien konzentrieren sich vor allem auf die zentralen Bezirke. Die Vielzahl an Linien in diesen Bezirken ermöglichen eine hohe Flexibilität durch viele Umsteigemöglichkeiten, wohingegen die Außenbezirke in der Regel nur von einzelnen Linien bedient werden.

Auf Ebene der Straßenbahn verkehren täglich 22 Linien mit 342 Fahrzeugen zwischen den 803 Haltestellen der Stadt. Hier fällt auf, dass die Straßenbahnlinien lediglich in den östlichen Bezirken verlaufen. Die Anbindung an das Straßenbahnnetz im übrigen Stadtgebiet ist nicht gegeben.

Im Gegensatz dazu bietet die Businfrastruktur eine flächendeckende Anbindung an das ÖPNV-Netz. Die 6.481 Bushaltestellen stehen gleichmäßig über das gesamte Stadtgebiet verteilt zur Verfügung. Ausgehend von den 154 Buslinien bildet die Businfrastruktur ein dichtes und weitreichendes Netz.

Die Infrastruktur der BVG wird durch 16 S-Bahnlinien ergänzt, die 166 S-Bahnhöfe bedienen und durch die VBB betrieben werden. Für die S-Bahnen stehen im Raum Berlin 335 Kilometer Schieneninfrastruktur zur Verfügung.


\subsubsection{Arbeitspaket X: Whitespot-Analyse}
