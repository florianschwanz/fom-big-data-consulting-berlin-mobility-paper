\newpage

\section{Ergebnispräsentation (Arbeitspaket 4)}
\label{ergebnispraesentation}

Die Analyseergebnisse werden in Form von interaktiven Karten und begleitenden Texten in Form einer Webapplikation dargestellt, welche auf Firebase Hosting deployed und unter der URL \url{https://berlin-mobility.web.app} erreichbar ist. Die Grundlage der Webapplikation bildet das quelloffene Webapplikationsframework Angular\footnote{https://angular.io/}, welches in Version 10 zum Einsatz kommt.

\img{app-01-landing-page}{app-01-landing-page}{width=0.5\textwidth}{Startseite der Webapplikation}

Die Webapplikation ist in zwei Bereiche unterteilt, welche im Folgenden beschrieben werden.

\begin{itemize}
    \item Die \emph{Story} stellt die Ergebnisse der Arbeitspakete 1-3 dar.
    \item Das \emph{Dashboard} ermöglicht dem Leser, in einer Karte interaktiv Ebenen ein- und auszublenden.
\end{itemize}

\subsection{Visualisierung der Analyseergebnisse}
\label{visualisierung_der_analyseergebnisse}

Das zentrale Konzept bei der Visualisierung der Analyseergebnisse ist die Verwendung von Scrollen als einziges Mittel der Navigation. Durch Scrollen kann sich der Leser durch die Abschnitte bewegen. Gleichzeitig werden durch das Scrollen Overlays auf Karten geändert und Animationen ausgelöst.  Um dem Leser eine Einführung in das Thema zu geben, werden zunächst die Hintergründe des Projekts beschrieben. Dabei werden sowohl das Berlin Mobilitätsgesetz, als auch die daran geäußerte Kritik ausgeführt.

Im zweiten Abschnitt werden dem Leser die Analyseergebnisse des Arbeitspakets 1 erläutert (siehe Kapitel~\ref{arbeitspaket_1_beschreibung_der_ist_situation}). Dabei wird eine Kombination aus Kartenmaterial, Overlays und erklärendem Text dargestellt, wobei sich die Overlays und Marker auf der Karte auf den jeweils dargestellten Beleittext beziehen. Die Interaktionsmöglichkeiten für den Leser sind bewusst gering gehalten. Lediglich die Marker auf einer Karte können durch das Bewegen der Maus über Textstellen, die sich auf einen Marker beziehen.

Im dritten Abschnitt wird im Anschluss anhand zweier Beispiele die Bedeutung und Funktionsweise von Isochronen erläutert. Eine animierte Karte mit entsprechender Legende vermittelt dem Leser anschaulich, wie sich Isochronen bei Veränderung des zeitlichen Parameters verändern. Ebenfalls wird ersichtlich, inwiefern sich Isochronen unterschiedlicher Verkehrsmittel unterscheiden.

Der vierte Abschnitt des Story-Bereichs stellt die Ergebnisse des Arbeitspakets 2 dar (siehe Kapitel~\ref{arbeitspaket_2_aufzeigen_von_schwachstellen_der_vorhandenen_infrastruktur}), welches sich mit den Schwachstellen des Berliner Verkehrsnetzes befasst. Um die Nachvollziehbarkeit zu gewährleisten, wird dem Leser zuvor die in Kapitel~\ref{verkehrsmittelspezifische_analyse} skizzierte Vorgehensweise beschrieben. Die Analyseergebnisse sind mittels \emph{Turf}\footnote{http://turfjs.org/docs/\#hexGrid} zu Hexagonen aggregiert und farblich kodiert. Gut angebundene Gebiete sind dabei grün hinterlegt, schlecht angebundene hingegen magenta.

Der Fokus des finalen Abschnitts liegt auf der Präsentation der Lösungsansätze (siehe Kapitel~\ref{arbeitspaket_3_ableiten_von_handlungsempfehlungen}). Die fünf Personas werden hierbei nacheinander beschrieben. Dabei werden die zum jeweiligen Szenario gehörigen Orte, Verkehrslinien und Haltestellen auf der Karte hervorgehoben. Ein dynamischer Zoom zentriert den Kartenausschnitt, welcher durch das jeweilige Szenario beschrieben ist.

\subsection{Interaktives Mobilitäts-Dashboard}
\label{interaktives_mobilitaets_dashboard}
