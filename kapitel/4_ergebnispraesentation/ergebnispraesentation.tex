\newpage

\section{Ergebnispräsentation (Arbeitspaket 4)}
\label{ergebnispraesentation}

Die Analyseergebnisse werden in Form von interaktiven Karten und begleitenden Texten mittels einer Webapplikation zur Verfügung gestellt. Die Webapplikation ist unter der URL \url{https://berlin-mobility.web.app} für die Öffentlichkeit zugänglich. Die Grundlage der Webapplikation bildet das quelloffene Webapplikationsframework Angular\footnote{https://angular.io/}, welches in Version~10 zum Einsatz kommt.

\img{app-landing-page}{app-landing-page}{width=14cm}{Startseite der Webapplikation}

Die Webapplikation ist in zwei Bereiche unterteilt, welche im Folgenden beschrieben werden.

\begin{itemize}
    \item Die \emph{Story} stellt die Ergebnisse der Arbeitspakete 1-3 dar.
    \item Das \emph{Dashboard} ermöglicht der Leser*in die Daten selbst zu erkunden, indem in einer Karte interaktiv Ebenen ein- und ausgeblendet werden können.
\end{itemize}

\subsection{Visualisierung der Analyseergebnisse}
\label{visualisierung_der_analyseergebnisse}

Innerhalb der \emph{Story} geht die Navigation ausschließlich von der Betätigung des Mausrads aus. Durch Scrollen kann sich die Leser*in durch die Abschnitte bewegen. Gleichzeitig werden durch das Scrollen Informationsebenen (sogenannte Overlays) auf den Karten geändert und Animationen ausgelöst. Um der Leser*in eine Einführung in das Thema zu geben, werden zunächst die Hintergründe des Projekts beschrieben. Dabei werden sowohl das Berlin Mobilitätsgesetz, als auch die daran geäußerte Kritik ausgeführt.

Im zweiten Abschnitt werden die Analyseergebnisse des Arbeitspakets 1 aufbereitet (siehe Kapitel~\ref{beschreibung_der_ist_situation_arbeitspaket_1}). Dabei wird eine Kombination aus Kartenmaterial, Overlays, Markern und erklärendem Text genutzt. Durch Scrollen kann sich die Leser*in im Text bewegen. Das Kartenmaterial sowie die Overlays und Marker passen sich dabei immer den sichtbaren textlichen Inhalten an. Die Interaktionsmöglichkeiten für die Leser*in sind bewusst gering gehalten. Neben der Möglichkkeit zu scrollen, können Leser*innen Marker, durch Bewegen der Maus über relevante Textstellen, auf dem Kartenmaterial ein- und ausblenden.

Im dritten Abschnitt wird anhand zweier Beispiele die Bedeutung und Funktionsweise von Isochronen erläutert. Eine animierte Karte mit entsprechender Legende vermittelt der Leser*in anschaulich, wie sich Isochronen bei Veränderung des zeitlichen Parameters verändern. Ebenfalls wird ersichtlich, dass sich Isochronen verschiedener Verkehrsmittel deutlich voneinander unterscheiden.

Der vierte Abschnitt der \emph{Story} stellt die Ergebnisse des Arbeitspakets 2 dar (siehe Kapitel~\ref{aufzeigen_von_schwachstellen_der_vorhandenen_infrastruktur_arbeitspaket_2}), und befasst sich dementsprechend  mit den Schwachstellen des Berliner Verkehrsnetzes.

Zu Beginn des Abschnitts wird die in Kapitel~\ref{verkehrsmittelspezifische_analyse} beschreibene Vorgehensweise zur Berechnung der Anbindungsqualität vorgestellt. Die Analyseergebnisse werden mittels \emph{Turf}\footnote{http://turfjs.org/docs/\#hexGrid} zu Hexagonen aggregiert und farblich kodiert. Gut angebundene Gebiete sind  grün hinterlegt, schlecht angebundene Gebiete sind magentafarben gekennzeichnet. Anhand dieser einfachen Farbkodierung sind die Analysergebnisse auch für Laien leicht zugänglich.

Der Fokus des finalen Abschnitts liegt auf der Präsentation der Lösungsansätze (siehe Kapitel~\ref{ableiten_von_handlungsempfehlungen_arbeitspaket_3}). Zunächst wird das Priorisierungsverfahren erörtet, welches zur Auswahl der fünf adressierten Orte genutz wird. Die Lösungsansätze werden anhand von fünf Personas beschrieben. Dabei werden die zur jeweiligen Persona gehörigen Orte, Verkehrslinien und Haltestellen auf der Karte hervorgehoben. Ein dynamischer Zoom zentriert den Kartenausschnitt, der ausgehend von der jeweiligen Persona beschrieben ist.

\subsection{Interaktives Mobilitäts-Dashboard}
\label{interaktives_mobilitaets_dashboard}

Um der Kritik gefühlter Intransparenz, die in Kapitel~\ref{problems} beschrieben ist, entgegenzuwirken, wurde ein interaktives Dashboard entwickelt. Dieses Dashboard ermöglicht es Bürger*innen sich auch ohne spezifisches Vorwissen mit den Daten und Erkenntnissen zu beschäftigen, die als Grundlage der Mobilitätsentscheidungen des Senats dienen.

Auf mehreren Verkehrsmittelspezifischen Unterseiten und einer Seite mit verkehrsmittelübergreifenden Information haben Bürger*innen, Entscheider*innen sowie Städte*planerinnen die Möglichkeit, über die Inhalte der \emph{Story} hinaus, Daten auf eigene Faust zu erkunden. Durch interaktive Elemente können Nutzer*innen beispielsweise die Verkehrssituation verschiedener Orte innerhalb der Stadt miteinander vergleichen.

Auf den folgenden Unterseiten haben Nutzer*innen die Möglichkeit direkt mit Mobiltiätsdaten zu interagieren:

\subsubsection{Öffentlicher Personennahverkehr}

Auf dieser Seite werden alle Berliner Bus- und Bahnverbindungen des öffentlichen Nahverkehrs innerhalb von Berlin visualisiert. Darüber hinaus können Nutzer*innen auf einzelnen Ebenen die Erreichbarkeit der Stadt mittels aller öffentlichen Verkehrsmittel ein- und ausblenden.

\img{app-dashboard-opnv}{app-dashboard-opnv}{width=14cm}{Interaktives Dashboard öffentlicher Personennahverkehr}

\subsubsection{Motorisierter Individualverkehr}

Auf dieser Seite können Nutzer*innen die Engstellen des Straßenverkehrs betrachten. Auf vier Kacheln werden die durchschnittlich tatsächlich gefahrenen Geschwindigkeiten, die Verteilung der maximal zulässigen Höchstgeschwindigkeiten innerhalb der Stadt, Stoßzeiten auf einer Zeitachse sowie die Erreichbarkeit mittels der 15-Minuten-Isochrone visualisiert.

\img{app-dashboard-miv}{app-dashboard-miv}{width=14cm}{Interaktives Dashboard motorisierter Individualverkehr}
\paragraph{Zukünftige Erweiterungen}

Um den Nutzer*innen der Webapplikation zukünftig einen noch breiteren sowie einfacheren Zugang zu Mobilitätsdaten zu ermögichen, werden weitere Dashboards entwickelt. In der Planung befindet sich beispielsweise ein Dashboard, dass sich mit dem geplante Tram-Netz beschäftigt, und ein Dashboard, dass auf den in Kapitel~\ref{mobwob_index} beschriebenen Mobilitätsindex eingeht.
